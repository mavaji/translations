% Default Compiler: txs:///xelatex
% Default Bibliography Tool: BibTex

\documentclass[12pt]{book}
\usepackage[x11names]{xcolor}
\usepackage{titlesec}
\usepackage[linktocpage=true,colorlinks,citecolor=blue,pagebackref=true]{hyperref}
\usepackage[top=30mm, bottom=30mm, left=30mm, right=30mm]{geometry}
\usepackage[utf8]{inputenc}
%\newcommand{\noun}[1]{\textit{\textcolor{black!60}{#1}}}
\newcommand{\noun}[1]{«{#1}»}

\usepackage{setspace}
%\onehalfspacing
\doublespacing

\usepackage{xepersian}
\settextfont[Scale=1.2]{IRNazanin}
\defpersianfont\mfo[Scale=1.2]{IRNazanin}
\setlatintextfont[Scale=1]{Doulos SIL}

%\titleformat{\chapter}
%{\normalfont\huge\bfseries}{\chaptertitlename\ \thechapter.}{20pt}{\huge}
%\renewcommand \chaptertitlename {اپیزود}
\renewcommand{\chaptername}{اپیزود}

\titleformat
{\chapter} % command
[display] % shape
{\bfseries\Large\itshape} % format
{اپیزود \ \thechapter} % label
{0.5ex} % sep
{
\rule{\textwidth}{1pt}
\vspace{1ex}
\centering
} % before-code
[
\vspace{-0.5ex}%
\rule{\textwidth}{0.3pt}
] % after-code

\begin{document}
    \title{رویایِ یِک مَردِ مُضحِک}
    \author{فئودور داستایفسکی\\
    ترجمهٔ وحید مواجی
    }
    \date{مرداد ۱۳۸۵}
    \frontmatter                            % only in book class (roman page #s)
    \maketitle                              % Print title page.
    \tableofcontents                        % Print table of contents
    \mainmatter


    \part{رویای یک مرد مضحک}
    \paragraph{}
    من یک آدمِ مضحک هستم. البته الآن به من میگن دیوانه. اگه به اندازهٔ قبل در نظرِ اونها مضحک نبودم، میشد گفت که به من ترفیع دادن. ولی الآن دیگه از این امر نمی‌رنجم، همشون برای من عزیزن حتی وقتی به من می‌خندن - و در واقع وقتی به من می‌خندن، بیشتر برا من عزیز می‌شن. منم می‌تونستم باهاشون بخندم - نه دقیقاً به خودم، بلکه به خاطر علاقه‌ای که به اونها دارم، البته اگه وقتی بهشون نگاه می‌کنم اینقدر غمگین نشم. غمگین برا اینکه اونها حقیقت‌و نمی‌دونن و من می‌دونم. آه، چقدر سخته که تنها کسی باشی که حقیقت رو می‌دونه! ولی اونها نمی‌فهمن. نخیر، نمی‌فهمن.

    \paragraph{}
    قبلنا، من احساس بدبختی می‌کردم، به خاطر اینکه مضحک به نظر می‌رسیدم. نه اینکه به نظر می‌رسیدم، بلکه واقعاً مضحک بودم. من همیشه مضحک بودم، و این‌و می‌دونستم، شاید، از لحظه‌ای که به دنیا اومدم. شاید از وقتی که هفت سالم بود می‌دونستم که مضحکم. بعدش رفتم مدرسه، تو دانشگاه درس خوندم، و آیا می‌دونین، هر چی بیشتر یاد می‌گرفتم، بهتر می‌فهمیدم که مضحکم. طوری که سرآخر هر چی بیشتر تو تمام علومی که در دانشگاه یاد گرفتم، عمیق می‌شدم، به این نتیجه می‌رسیدم که این علوم فقط به این خاطر وجود دارن که به من ثابت کنن مضحکم. در مورد زندگی هم قضیه عینهو مثل علم و دانش بود. سال به سال، آگاهی من نسبت به مضحک بودنم در رابطه با آدم‌ها، رشد کرده و قوی‌تر می‌شد. همه همیشه به من می‌خندیدن. ولی هیچ کدوم از اونها نمی‌دونست یا حدس نمی‌زد که اگه یه نفر تو دنیا وجود داشته باشه که بهتر از بقیه بدونه من مضحکم، اون یه نفر خودم هستم، و چیزی که بیشتر از همه من‌و رنج می‌داد این بود که اونها اینو نمی‌دونستن. ولی تقصیر خودم بود؛ به قدری مغرور بودم که هیچ چیز باعث نشد تا این واقعیت رو به همه بگم. با گذشت سالیان، این غرور در من رشد کرد؛ و اگه می‌شد این اجازه رو به خودم می‌دادم که به همه اعتراف کنم آدم مضحکی هستم، مطمئناً بعدازظهرِ همون روز، مغز خودم‌و داغون می‌کردم. آه، چقدر در اَوانِ جوانی از ترس این که یه وقت کم بیارم و به همکلاسی‌هام اعتراف کنم، رنج بردم. ولی وقتی مَرد شدم، به دلایل نامعلومی آرام‌تر شدم، با این که هر سال به ویژگی‌های مزخرف خودم بیشتر واقف می‌شدم. به این خاطر گفتم «نامعلوم»، که هنوز که هنوزه نمی‌دونم چرا. شاید به دلیلِ بدبختیِ عظیمی بود که به خاطر «چیزی» در روحِ من در حال رشد بود و بیشتر از هر چیزِ دیگه‌ای عواقبش متوجه من بود: اون «چیز»، متقاعد شدن به این بود که هیچ چیز تو دنیا اهمیت نداره. مدت‌های طولانی یه چیزایی راجع به این قضیه می‌دونستم ولی ادراک کامل اون، به طور تقریباً تصادفی، پارسال اتفاق افتاد. تصادفاً دریافتم برای من علی‌السویه است که دنیا وجود داشته باشه یا اصلاً هیچ چیز، هیچ وقت وجود نداشته: با تمام وجودم احساس می‌کردم که چیزی وجود نداره. اولش تصور می‌کردم در گذشته، چیزای زیادی وجود داشته، ولی بعدش حدس زدم که هیچ وقت، در گذشته هم چیزی وجود نداشته بلکه به دلایلی من این طور تصور می‌کردم که چیزایی وجود داره. کم‌کم حدس زدم که در آینده هم چیزی وجود نخواهد داشت. اونوقت بود که دیگه از دست مردم عصبانی نبودم و حتی تقریباً توجهی هم به اونها نداشتم. در واقع این بی‌توجهی خودش‌و تو کوچکترین چیزها نشون می‌داد: مثلاً تنه‌زدن به مردم تو خیابون، برام عادی شده بود. و نه به این خاطر که تو افکار خودم غرق بودم: چی داشتم که بهش فکر کنم؟ اون موقع اصلاً فکرکردن‌و کنار گذاشته بودم؛ هیچ چیز برام مهم نبود. ایکاش حداقل مشکلاتم رو حل کرده بودم! آه، هیچ کدوم از اونها رو راست و ریست نکرده بودم و چقدرم زیاد بودن! ولی دیگه به هیچ چیز توجه نکردم، و کلِ مشکلاتم پر زدن و رفتن.

    \paragraph{}
    و بعد از این کار بود که حقیقت رو پیدا کردم. حقیقت رو نوامبرِ سالِ قبل فهمیدم - سه نوامبر، اگه بخوام دقیق بگم - و همه چیز رو خوب به خاطر دارم. یه غروبِ غم‌انگیز بود، یکی از غمگین‌ترین غروب‌هایی که می‌تونه وجود داشته باشه. حدود ساعت یازده داشتم به خونه برمی‌گشتم، و یادم می‌آد که همون موقع فکر کردم هیچ غروبی غم‌انگیزتر از این نمی‌تونه وجود داشته باشه. حتی از نظر عوامل طبیعی. تمام روز بارون می‌اومد و بارونش هم سرد، تیره و ترسناک بود و اونطور که به یادم می‌آد با کینه و بغض نسبت به بشریت می‌بارید. تصادفاً بین ده و یازده، بند اومد و پشت سرش رطوبت وحشتناکی همه جا رو فراگرفت. رطوبتی که سردتر و مرطوب‌تر از بارون بود، و یه جور مِه و بخار از هر چیزی بلند می‌شد، از هر سنگ خیابون و از هر کوچه‌پس‌کوچه‌ای تا اون جا که چشم کار می‌کرد. ناگهان فکری به ذهنم خطور کرد، که اگه همهٔ چراغ‌های خیابون خاموش می‌شدن، غمگینیِ خیابون کمتر می‌شد چون گاز باعث می‌شه قلب آدم غمگین‌تر بشه چرا که تماماً اون‌و روشن می‌کنه. اون روز ناهارِ خیلی کمی خورده بودم و بعدازظهرم‌و با یه مهندس گذرونده بودم و دو تا رفیق دیگه هم با ما بودن. من ساکت نشستم - فکر می‌کنم خسته‌شون کردم. اونها در مورد یه چیزِ هیجان‌انگیز صحبت می‌کردن و یه دفعه هیجان‌زده شدن. ولی اونها واقعاً توجه نمی‌کردن، می‌تونستم حرفشون‌و بفهمم و فقط تظاهر کنم که هیجان‌زده شده‌ام. یهو منم به همون شدت باهاشون حرف زدم. من گفتم: «رفقا، شما به هیچ چیز توجه نمی‌کنین». اونها دلخور نشدن ولی به من خندیدن. به این خاطر بود که حالت سرزنش تو حرفم نبود چون که اصلاً برام مهم نبود. اونها فهمیدن که برام مهم نیست و این امر سرگرمشون کرد.

    \paragraph{}
    همونطور که داشتم به لامپ‌های گازی خیابون فکر می‌کردم به آسمون نگاه کردم. آسمون، وحشتناک تاریک بود، ولی می‌شد به وضوح ابرهای پاره‌پاره و بین اونها تیکه‌های سیاهِ بی‌انتها رو دید. ناگهان تو یکی از این تیکه‌های سیاه، یه ستاره دیدم و تعمداً بهش نگاه کردم. به این خاطر بود که اون ستاره یه ایده به من داد: تصمیم گرفتم که اون شب خودم‌و بکشم. من دو ماه قبلش خیلی محکم این تصمیم‌و گرفته بودم و اونقدر بدبخت بودم که همون روز یه طپانچهٔ عالی خریدم و پُرِش کردم. ولی دو ماه گذشت و اون طپانچه همینجور تو کِشوی من افتاده بود؛ به قدری بی‌تفاوت بودم که منتظر بودم یه فرصت به دست بیارم که اینقدر بی‌تفاوت نباشم - چرا، نمی‌دونم. و به مدت دو ماه، هر شب که به خونه میومدم فکر می‌کردم که خودم‌و می‌کشم. برا رسیدن لحظهٔ مناسب ثانیه‌شماری می‌کردم. و حالا این ستاره باعث شد یه فکر به ذهنم برسه. به خودم تلقین کردم که حتماً امشب وقتشه. و چرا اون ستاره باعث این فکر شد، نمی‌دونم.

    \paragraph{}
    و درست موقعی که داشتم به آسمون نگاه می‌کردم، یه دختر کوچولو آرنجِ من‌و گرفت. خیابون خالی بود و به ندرت کسی دیده می‌شد. کمی دورتر، یه راننده تاکسی تو تاکسیش خوابیده بود. یه بچهٔ هشت‌ساله بود که به سرش لچک بسته بود و به غیر از یه لباس کوچیک که خیسِ بارون شده بود، هیچی تنش نبود، ولی من متوجه کفش‌های مندرس خیسش شدم و الآن به خاطر میارمشون. خیلی نظرم‌و جلب کرده بودن. اون یه دفعه آرنجِ من‌و گرفت و صِدام کرد. گریه نمی‌کرد، ولی به طرزی غیرعادی کلماتی را فریاد می‌زد که نمی‌تونست درست اداشون کنه چون تمام بدنش می‌لرزید. از چیزی می‌ترسید و داد زد «مامان، مامان!» روم‌و ازش برگردوندم، هیچی نگفتم و رفتم؛ ولی اون دوید، به من آویزون شد و اون لحنی که تو بچه‌های ترسیده علامتِ ناامیدیه، تو صداش بود. من اون صدا رو می‌شناسم. با این که درست صحبت نکرد، فهمیدم که مادرش داره می‌میره یا یه چیز تو همین مایه‌ها داره براشون اتفاق می‌افته و اون دویده که یکی رو خبر کنه، یه چیزی پیدا کنه که به مادرش کمک کنه. من باهاش نرفتم؛ برعکس انگیزهٔ شدیدی داشتم که اون‌و از خودم برونم. اول بهش گفتم که بِرِه پیشِ پلیس. ولی دستاش‌و به هم گره کرد، پشت سرم دوید و هق‌هق می‌کرد و نفس‌نفس می‌زد و دست از سرم برنمی‌داشت. اونوقت پاهام‌و به زمین کوبیدم و رو سرش داد زدم. التماس می‌کرد «آقا! آقا! ...» ولی یهو وِلَم کرد و با کلّه تو خیابون دوید. یه عابرِ دیگه اونجا پیداش شده بود و اون دختر از طرفِ من پرواز کرد به طرفِ اون.

    \paragraph{}
    رفتم بالا به اتاقم که تو طبقهٔ پنجم بود. من یه اتاق تو یه آپارتمان دارم که توش مستأجرهای دیگه هم زندگی می‌کنن. اتاقم کوچیک و محقره، و یه پنجرهٔ زیرشیروونی به شکل نیم‌دایره داره. من یه مبل دارم که چرمش آمریکاییه، یه میز دارم که کتابام روشه، دو تا صندلی به اضافه یه صندلیِ دسته‌دار هم دارم که علی‌رغم قدیمی بودنِ، شکلِ از مُد افتادَش قشنگه. نشستم، شمعی روشن کردم و شروع کردم به فکرکردن. تو اتاقِ بغلی، که اون طرفِ تیغهٔ پارتیشن بود، یه تیمارستانِ واقعی وجود داشت. این دیوونه‌خونه از سه روز پیش شروع به کار کرده بود. یه سروانِ بازنشسته اون جا زندگی می‌کرد و یه دوجین ملاقات‌کننده داشت، آدم‌هایی که شهرت خوبی نداشتن و می‌نشستن ودکا می‌خوردن و با کارت‌های کهنه، استاس بازی می‌کردن. شب قبلش دعواشون شده بود و باید بگم که دوتاشون برای مدت زیادی داشتن گیس و گیس‌کشی می‌کردن. خانمِ صاحبخانه می‌خواست اعتراض کنه، ولی مثل سگ از سروان می‌ترسید. تو اون طبقه فقط یه مستأجرِ دیگه زندگی می‌کرد، یه خانم سرهنگ ریزه‌میزهٔ لاغر که یه زمانی گذرش به پترزبورگ افتاده بود و سه تا بَچّش مریض شده بودن و از اون موقع تا حالا اومده بود کرایه‌نشینی. هم خودش و هم بچه‌هاش تا حدِ مرگ از سروان می‌ترسیدن و تمامِ شب می‌لرزیدن و به خودشون می‌پیچیدن و بچه کوچیکتره از ترسِ دعوای آقایون دچار غش شد. این سروان، از جایی شنیدم که جلوی مردم‌و تو «نوسکی پروسپکت» می‌گیره و گدایی می‌کنه. دیگه تو خدمت نظام راهش نمی‌دن ولی عجیبه که (به همین خاطر دارم این‌و می‌گم) که کُلِّ این یه ماهی که سروان این جا بود، رفتارش اصلاً منو آزار نداد. البته از اولش از برخورد با اون احتراز می‌کردم و اون هم از اولش حوصلهٔ من‌و نداشت. ولی من هیچ وقت برام مهم نبود که چقدر اونها اونورِ پارتیشن عربده می‌کشن یا چند نفرشون اون جا چتر انداختن. کل شب رو بیدار می‌شینم و اون قدر اونها رو فراموش می‌کنم که حتی صداشونم نمی‌شنوم. تا صبح بیدار می‌مونم و این کار رو به مدت یه ساله که انجام می‌دم. کل شب رو روی صندلیِ دسته‌دارم بیدار می‌شینم و هیچ کاری نمی‌کنم. فقط روزا مطالعه می‌کنم. من می‌شینم - حتی فکر هم نمی‌کنم؛ یه جور فکرایی تو مغزم ول می‌چرخن و منم آزادشون می‌ذارم که هر جور دلشون خواست این‌ور اون‌ور برن. هر شب یه شمعِ کامل می‌سوزه. اون شب، ساکت روی میز نشستم، طپانچه رو برداشتم و گذاشتم جلوی خودم. وقتی این کارو کردم یادم میاد از خودم پرسیدم: «این جوریاس؟» و با اعتقاد راسخ جواب دادم: «این جوریاس دیگه.» یعنی اینکه باید به خودم شلیک می‌کردم. مطمئن بودم که اون شب باید به خودم شلیک کنم ولی نمی‌دونستم که قبلش چقدر باید پشت میز بشینم و صبر کنم. و شکی نیست که اگه به خاطر اون دختر کوچولو نبود، همون شب به خودم شلیک می‌کردم.

    \paragraph{}
    دیدین با اینکه هیچ چیز برام مهم نبود، ولی چه طور رنج می‌کشیدم. اگه کسی بِهِم می‌چسبید، اعصابم‌و خورد می‌کرد. از نظر روحی هم وضع به همین منوال بود: اگه موضوع سوزناکی پیش می‌اومد، درست مثل قدیما که چیزایی تو زندگی وجود داشت که برام مهم بود، احساس همدردی می‌کردم. اون روز  غروب هم احساس همدردی بِهِم دست داد. مطمئناً من باید به یه بچه کمک می‌کردم. پس چرا به اون دختر کوچولو کمک نکردم؟ به خاطر فکری که اون لحظه به ذهنم رسید: وقتی داشت من‌و صدا می‌کرد و می‌کشید، ناگهان سؤالی برام پیش اومد و نتونستم برای خودم تجزیه‌تحلیلش کنم. اون سؤال، خیلی مزخرف بود ولی من‌و آزار می‌داد. من با این فکر که اگه قراره امشب دَخلِ خودم‌و بیارم، پس چیزی تو زندگی نباید دیگه برام مهم باشه، آزرده شدم. چرا اینجوری بود که یِهو هیچ درد غیرعادی‌ای احساس نکردم و خیلی راحت سرِ جای خودم ایستادم؟ در واقع من بهتر از این نمی‌تونم احساساتِ گذرای خودم‌و تو اون لحظه به شما انتقال بدم ولی این احساسات تو خونه، وقتی که پشت میز نشسته بودم هم ادامه داشتن و به قدری عصبانی بودم که مدت‌ها بود سابقه نداشت. افکار، یکی بعد از دیگری میومدن و می‌رفتن. به وضوح دیدم تا موقعی که من هنوز آدم بودم و نه یه چیزِ پوچ، زنده بودم و می‌تونستم رنج بکشم، عصبانی بِشَم و از کارهام خجالت بکشم. بسیار خوب. ولی اگه قراره تا دو ساعت دیگه خودم‌و بُکُشم، پس دیگه من با اون دختر کوچولو چی کار دارم و دیگه خجالت کشیدن یا هر کار دیگه‌ای تو این دنیا به من چه ربطی داره؟ من قراره نابود بشم، کاملاً نابود. و آیا واقعاً می‌تونه راست باشه که خودآگاهی‌ای که می‌خوام یه دفعه از بین بِبَرَمِش و به تَبَعِش همه چیزِ دیگه هم از بین می‌ره،  اصلاً رو احساسِ ترحم من نسبت به بچه یا احساس شرمندگی من بعد از یه کار کثیف تأثیری نداشته باشه؟ من پاهام‌و به زمین کوبیدم و سَرِ یه بچهٔ بیچاره داد کشیدم انگار که می‌خواستم بگم: نه تنها احساس ترحم نمی‌کنم، بلکه حتی اگه رفتارم کثیف و غیرانسانی باشه آزادم هر جور که دلم می‌خواد باشم چرا که تا دو ساعت دیگه همه چیز محو خواهد شد. باور می‌کنین که به همین خاطر بود که سَرِش داد زدم؟ الآن دیگه کاملاً از این قضیه مطمئنم. برام مثل روز روشن بود که زندگی و دنیا الآن یه جورایی به من وابسته است. حتی می‌تونم بگم که دنیا انگار فقط برای من ساخته شده بود: اگه من به خودم شلیک کنم، دیگه دنیایی برای من وجود نخواهد داشت. دیگه در این مورد بحث نمی‌کنم که وقتی من نباشم، ممکنه برای بقیه هم همه چیز نابود بشه و این که به محضِ محو شدن خودآگاهیِ من، کلِ دنیا هم به عنوان یه قسمت از خودآگاهیِ من بخار میشه و می‌پره مثل یه شبح. چون ممکنه همهٔ این دنیا و همهٔ این آدم‌ها، فقط خودِ خودِ من باشن. یادم میاد که نشستم و فکر کردم، همهٔ این سوال‌هایی رو که پشت سر هم می‌اومدن می‌پیچوندم و به یه چیزِ دیگه فکر می‌کردم.

    \paragraph{}
    مثلاً یه دفعه یه فکرِ عجیب به ذهنم رسید، که اگه قبلاً رو ماه یا مریخ زندگی می‌کردم و اونجا مرتکب ننگین‌ترین و شرم‌آور‌ترین اَعمال می‌شدم و دچارِ چنان خجالت و رسوایی‌ای می‌شدم که فقط توی رویاها و کابوس‌ها، میشه اون‌و درک کرد و فهمید، و اگه بعدش می‌دیدم که رو زمین هستم و قادر بودم که خاطرات آنچه که روی سیاراتِ دیگر انجام داده‌ام رو حفظ کنم و درعین‌حال می‌دونستم که تحت هیچ شرایطی به اون جا برنخواهم گشت و از زمین به ماه نگاه می‌کردم، آیا باید دلواپس می‌شدم یا نه؟ آیا باید به خاطر کاری که انجام داده بودم شرمگین می‌شدم یا نه؟ این سؤالات بیخودی و چرت و پرت بودن، چرا که طپانچه جلوی من قرار داشت و من با تمام وجودم می‌دونستم که قطعاً اون کار اتفاق میفته، ولی اون سؤال‌ها من‌و هیجان‌زده کردن و خشمگین شدم. حالا باید قبل از مردن، یه چیزی رو راست و ریست می‌کردم. در این اثنا، سر و صدا تو اتاقِ سروان فروکش کرد: اونها بازیشون‌و تموم کرده بودن، داشتن آمادهٔ خواب می‌شدن و با این حال غرغر می‌کردن و با خماری به دعواشون خاتمه می‌دادن. در اون لحظه، ناگهان تو صندلیِ خودم پشتِ میز، خوابم گرفت - چیزی که قبلاً هیچ وقت برام پیش نیومده بود. تقریباً به طور ناخودآگاه به خواب رفتم.

    \paragraph{}
    همونطور که همه‌مون می‌دونیم، رویاها چیزای عجیب‌غریبی هستن: بعضی قسمت‌ها با وضوحِ خیره‌کننده‌ای دیده می‌شن، با جزئیاتی که با پرداختِ استادانه‌ای، جواهرکاری شده‌اند در حالی که بقیهٔ قسمت‌ها رو، آدم سرسری نگاه می‌کنه بدون این که اصلاً به چیزی در زمان و مکان توجه کنه. به نظر می‌رسه که رویاها نه با منطق که با آرزو، نه با مغز که با قلب به وجود میان، و با این وجود چه حقه‌های پیچیده‌ای که بعضی مواقع، منطق به رویا نزده و چه چیزهای کاملاً غیرقابل‌فهمی که در رویا اتفاق نمیفته! مثلاً برادرِ من پنج سال پیش مُرد. من بعضی موقع‌ها خوابش‌و می‌بینم؛ اون تو کارای من شرکت می‌کنه، ما خیلی به هم دلبستگی داریم و با این حال در کلِ مدتِ خوابم می‌دونم و به یاد میارم که برادرم مُرده و زیرِ خاکه. چه جوریه که با این که اون مُرده ولی کنارِ منه و داریم با هم کار می‌کنیم، اصلاً منو متعجب نمی‌کنه. چرا منطقِ من کاملاً همچین چیزی رو قبول می‌کنه؟ ولی بسه. من می‌خوام رویام‌و تعریف کنم. بله، من یه رویا دیدم، رویای سومِ نوامبر من. الآن مردم من‌و دست میندازن، به من میگن که فقط یه رویا بوده. ولی چه فرقی می‌کنه که یه رویا بوده یا واقعیت، اگر که اون رویا حقیقت رو برای من آشکار کرده باشه؟

    \paragraph{}
    اگه یه زمانی، کسی حقیقت رو تشخیص بده و اون‌و ببینه، می‌دونه که حقیقت همینه و چیز دیگه‌ای نمی‌تونه باشه، چه خواب باشه چه بیدار. حالا فرض کنید که رویا بوده، دمش گرم، ولی اون زندگیِ واقعی‌ای که شما اونقدر روش تکیه می‌کنین، من می‌خواستم با خودکشی نابودش کنم و رویای من، رویای من، آه - اون زندگیِ متفاوتی رو برای من آشکار کرد، زندگی از نو، باشکوه و پر از قدرت. گوش بدین.

    \paragraph{}
    گفتم که بی‌اختیار به خواب رفتم و حتی به نظر می‌رسید که هنوز هم دارم به همون چیزها فکر می‌کنم. ناگهان خواب دیدم که طپانچه رو برداشتم و مستقیم به سمتِ قلبم نشونه گرفتم - قلبم و نه مغزم؛ در حالی که قبلاً تصمیم گرفته بودم مغزم‌و از طرفِ نیمکرهٔ راست داغون کنم. بعد از نشونه‌گیری به سمتِ سینه‌ام، یکی دو ثانیه صبر کردم و سپس شمعم، میزم و دیوارِ جلوی من شروع کردن به پیچ و تاب خوردن. به سرعت ماشه رو چکوندم.

    \paragraph{}
    توی رویاها، بعضی مواقع از ارتفاع پرت میشین، یا خنجر می‌خورین یا یکی شما رو می‌زنه، ولی هیچ وقت احساس درد نمی‌کنین مگر این که واقعاً خودتون‌و به تختخواب کوبیده باشین که در این صورت دردتون میاد و از اون درد بیدار میشین. تو رویای من هم همینطوری بود. من هیچ دردی احساس نکردم، ولی به نظر می‌رسید که انگاری با شلیک من، تمام امعاء و احشاء درونی‌ام دارن تکون می‌خورن و یهو همه چیز تیره و تار شد، و به طرز وحشتناکی دور و بَرَم به سیاهی رفت. انگار که کور شده بودم و احساسِ بی‌حسی می‌کردم و به پشت، روی یه چیزِ سفت دراز کشیده بودم؛ چیزی ندیدم، و کوچیک‌ترین حرکتی نمی‌تونستم بکنم. دور و برم، چند نفر داشتن راه می‌رفتن و فریاد می‌کشیدن، سروان عربده می‌زد، خانم صاحبخانه جیغ می‌زد - و ناگهان یه وقفهٔ دیگه و بعدش داشتن منو توی یه تابوتِ بسته حمل می‌کردن. و احساس می‌کردم که چقدر تابوت تکون می‌خوره و به این تکون خوردنها دقیق شدم، و برای اولین بار مغزم جرقه زد که نکنه من مرده‌ام، کاملاً مُرده. این مسأله رو فهمیدم و شکی بهش نداشتم. نه می‌تونستم ببینم، نه می‌تونستم تکون بخورم و با این حال داشتم احساس و فکر می‌کردم. ولی خیلی زود خودم‌و با شرایط وفق دادم و همونطور که معمولاً تو رویا اتفاق می‌افته، بی چون و چرا، همه چیزو قبول کردم.

    \paragraph{}
    و حالا زیر خاک بودم. هَمَشون رفتن؛ من تنها موندم، کاملاً تنها. تکون نخوردم. قبلاً هر وقت تصور دفن شدن رو می‌کردم، احساسم نسبت به قبر، یه جای مرطوب و سرد بود. بنابراین احساس کردم که خیلی سردمه، مخصوصاً نوکِ انگشت‌های پام، ولی چیز دیگه‌ای احساس نکردم.

    \paragraph{}
    همین طور دراز کشیده بودم و جالبه که منتظر هیچ چیز نبودم، بی چون و چرا قبول کرده بودم که یه مُرده نباید منتظر چیزی باشه. ولی اونجا مرطوب بود. من نمی‌دونم چه مدتی گذشت - یه ساعت یا چند روز یا چند سال. ولی یه دفعه، یه قطره آب چکید روی چشمِ چَپَم که بسته بود، اون قطره راهش‌و از شکاف تابوت پیدا کرده بود. یه دقیقه بعد، دومی هم چکید. یه دقیقه بعدش، سومی هم چکید - و به همین ترتیب، هر یه دقیقه یه قطره می‌چکید. بارقه‌ای از خشم عمیق و ناگهانی در قلبم به وجود اومد و در اون بارقه، سوزشی از دردِ طبیعی حس کردم. با خودم گفتم «این جای زخمه، زخمِ اون گلوله...» و هر دقیقه، یه قطره روی پِلکِ بسته‌ام می‌چکید. و کاملاً ناگهانی، نه با صِدام، بلکه با تمامِ وجودم، اون قدرتی رو که مسئول تمام این بلاهایی بود که داشت به سرم میومد، صدا کردم:

    «هر کی که می‌خوای باش، اگه وجود داری، و اگه چیزی معقول‌تر از این چیزایی که داره اینجا اتفاق میفته وجود داره، قربون دستت، بفرستش اینجا. ولی اگه داری با این آخِرَتِ مهیب و پوچ و مزخرف، از خودکشیِ احمقانهٔ من انتقام می‌گیری، بذار بهت بگم که هیچ شکنجه‌ای مثل تحقیر و تمسخری نیست که دارم با زبونِ بسته حسش می‌کنم، هر چند یه میلیون سال من‌و شکنجه کنی».

    \paragraph{}
    من این شکوائیه رو اقامه کردم و آرامش خودم‌و حفظ کردم. یه دقیقه سکوتِ کامل برقرار بود و بعدش یه قطرهٔ دیگه چکید، ولی با اطمینانی بی‌نهایت محکم فهمیدم که بلافاصله همه چیز تغییر خواهد کرد. و یهو زیر قبرم شکافته شد یعنی نمی‌دونم باز شد یا من تازه متوجِهِش شدم، ولی یه موجودِ تیره و ناشناخته من‌و گرفت و رفتیم به فضا. من یهویی بیناییِ خودم‌و به دست آوردم. نصفه‌شب بود، و هیچ وقت، هیچ وقت، چنین تاریکی‌ای ندیده بودم. ما داشتیم خیلی دور از زمین، تو فضا پرواز می‌کردیم. از اون موجودی که من‌و گرفته بود، سؤالی نپرسیدم؛ غرورِ خودم‌و حفظ کردم و منتظر موندم. به خودم اطمینان دادم که نمی‌ترسم. نمی‌دونم چه مدتی پرواز کردیم، نمی‌تونم تصور کنم؛ جوری اتفاق افتاد که همیشه تو خواب اتفاق میفته، وقتی که  فضا و زمان و قوانینِ تفکر و وجود رو نادیده می‌گیرین و فقط چیزهایی رو می‌بینین که مشتاقش هستین. یادم میاد که ناغافل، یه ستاره تو سیاهی دیدم. بی‌ملاحظه پرسیدم: «این صورت فلکی شِعرای یَمانیه؟»، چرا که با خودم عهد بسته بودم سؤال نپرسم.

    \paragraph{}
    موجودی که داشت منو حمل می‌کرد جواب داد: «نه، این اون ستاره ایه که اون شب موقع برگشتن به خونه، بین ابرها دیدی».

    \paragraph{}
    فهمیدم که چیزی مثل صورتِ آدمیزاد داره. عجیبه که اون موجود رو دوست نداشتم، در واقع نفرت شدیدی نسبت بِهِش داشتم. من انتظار عَدَمِ مطلق رو داشتم و به همین خاطر یه گلوله تو قلب خودم خالی کرده بودم. ولی حالا اینجا تو دست‌های موجودی بودم که البته آدم نبود ولی بالاخره زنده بود و وجود داشت. با اون حماقت عجیبی که تو خواب وجود داره با خودم فکر کردم: «که اینطور، پس زندگی پس از مرگ هم وجود داره». ولی در اعماق قلبم، تغییری حاصل نشد. فکر کردم: «و اگه مجبور باشم دوباره وجود داشته باشم و یه بارِ دیگه تحت سیطرهٔ قدرتی مقاومت‌ناپذیر زندگی کنم، نه مغلوب خواهم شد و نه تحقیر».

    \paragraph{}
    در حالیکه نمی‌تونستم از سؤال تحقیرآمیزی که مُتِضَمِنِ اعتراف بود خودداری کنم و احساس می‌کردم که احساسِ حقارتِ من با سوزن به قلبم می‌زنه، ناگهان به همراهِ خودم گفتم: «می‌دونی که من اَزَت ترسیدم و برای این کار، من‌و تحقیر کن». اون به سؤال من جواب نداد ولی یه دفعه حس کردم که نه تنها داره من‌و تحقیر می‌کنه بلکه داره به من می‌خنده و هیچ دلسوزی نسبت به من نداره و اینکه سفرِ ما، هدفی ناشناخته و اسرارآمیز داره که برای هیچ کس به غیر از من مهم نیست. ترس داشت در قلب من رشد می‌کرد. یه چیزی داشت بی‌صدا و دردناک از طرفِ همراهِ خاموشِ من، با من ارتباط برقرار می‌کرد و به تمامِ وجودم رخنه می‌کرد. ما داشتیم در فضای تاریک و ناشناخته پرواز می‌کردیم. برای مدتی دیدم‌و نسبت به صُوَرِ فلکی‌ای که برام آشنا بودن، از دست دادم. می‌دونستم که در فضاهای بیکران، ستارگانی وجود دارن که نورشون هزاران یا میلیون‌ها سال طول می‌کشه تا به زمین برسه. شاید داشتیم اون موقع در همین فضاها پرواز می‌کردیم. با اضطراب وحشتناکی که قلبم‌و آزار می‌داد، منتظر چیزی بودم. و ناگهان با احساسی آشنا که من‌و به اعماق بُرد، به خودم اومدم: یه دفعه متوجه خورشیدِ خودمون شدم! می‌دونستم که این نمی‌تونه خورشیدِ خودمون باشه، همون که به زمین، زندگی می‌بخشید، و این که در فاصله‌ای بی‌نهایت دور از خورشیدِ خودمون بودیم ولی به دلایلی نامعلوم با تمامِ وجودم فهمیدم که این یه خورشیده دقیقاً مثل خورشید خودمون، یه المثنی از اون. یه حس شیرین و هیجان‌آور با خوشیِ زیادی در قلبم طنین‌انداز شد: قدرتِ همسانی از نورِ مشابهی، نوری به من داد که در قلبم طنین‌انداز شد و اون‌و از خواب بیدار کرد، و برای اولین بار بعد از موقعی که تو قبر بودم، حسی از زندگی به من دست داد، زندگیِ گذشته‌ای که داشتم.

    \paragraph{}
    فریاد زدم: «ولی اگه این خورشید باشه، اگه دقیقاً عینِ خورشیدِ خودمون باشه، پس زمین کجاست؟»

    \paragraph{}
    و همراهِ من به ستاره‌ای دور اشاره کرد که با نورِ سبزی مثل زمرد، چشمک می‌زد. داشتیم مستقیماً به طرفِ اون پرواز می‌کردیم.

    \paragraph{}
    با لرزشی ناشی از عشقی مقاومت‌ناپذیر و وجدآمیز نسبت به زمینِ قدیمی‌ای که اون‌و ترک کرده بودم، فریاد زدم: «آیا چنین تکرارهایی در جهان ممکنه؟ آیا این می‌تونه قانونِ طبیعت باشه؟... و اگه زمینی اونجا باشه، می‌تونه عیناً مثلِ زمینِ خودمون باشه... کاملاً مشابه، به همون بدبختی و غمگینی و در عین حال عزیز و محبوبِ همه، و آیا می‌تونه در ناسپاس‌ترین فرزندانِ خود، همان عشقِ تندی رو که به زمینِ خودمون احساس می‌کنیم، نسبت به خودش به وجود بیاره؟» تصویرِ کودکِ بیچاره‌ای که از خودم روندمش، از ذهنم گذشت.

    \paragraph{}
    همراهِ من جواب داد: «همه چیزو خواهی فهمید». یه جور اندوه تو صداش بود.

    \paragraph{}
    ولی ما داشتیم به سرعت به اون سیاره نزدیک می‌شدیم. به تدریج، چیزها رو بهتر می‌دیدم؛ تقریباً می‌تونستم اقیانوس‌ها و  شکلِ قارهٔ اروپا رو تشخیص بدم؛ و ناگهان یه حسِ حسادتِ بزرگ و مقدس در قلب من شعله‌ور شد.

    \paragraph{}
    «چه جوری میشه همه چیز تکرار شده باشه و برای چی؟ من عاشق اون زمینی هستم که تَرکِش کردم، که موقع ناسپاسی، با خونم لکه‌دارش کردم، که با یه گلوله تو قلبم به زندگی‌ام خاتمه دادم و فقط هم عاشق اونم. ولی هرگز، هرگز از عشق خودم نسبت به زمین دست برنداشتم و شاید در اون شبِ کذایی که تَرکِش کردم، از همیشه بیشتر دوستش داشتم. آیا در این زمینِ جدید هم رنج وجود داره؟ در زمینِ خودمون فقط می‌تونیم با رنج بردن و از طریقِ رنج بردن، عشق بورزیم. جورِ دیگه‌ای نمی‌تونیم عاشق باشیم، و نوعِ دیگه‌ای از عشق رو نمی‌شناسیم. من برای عاشق بودن می‌خوام رنج ببرم. من مشتاق و تشنهٔ لحظه‌ای هستم که با اشک‌های خودم، زمینی رو که ترکش کردم ببوسم و زندگی روی زمین دیگه‌ای رو نه می‌خوام و نه می‌پذیرم.»

    \paragraph{}
    ولی همراهم، من‌و ترک کرده بود. ناگهان، بدون اینکه متوجه باشم چه جوری، در نورِ روشنِ یه روزِ آفتابی با لطافتی بهشتی، دیدم که رو اون یکی زمین هستم. مطمئنم که رو یکی از جزیره‌هایی که رو زمینِ خودمون، مجمع‌الجزایر یونان رو تشکیل میدن، یا در ساحلِ خشکی‌ای که روبروی اون مجمع‌الجزایر هست، ایستاده بودم. آه، همه چیز عیناً مثل چیزای خودمون بود، فقط به نظر می‌رسید که همه چیز، با شادی می‌درخشه، و دارای شکوه و جلالِ فتحی بزرگ و مقدس هست. دریای دلنواز، سبز مثل زمرد به نرمی روی ساحل می‌لغزید و با عشقی آشکار و تقریباً هوشیار اون‌و می‌بوسید. درختانِ بلند و دوست‌داشتنی با تمامِ شکوهِ جوانی‌شون،  قد برافراشته بودن و مطمئنم که برگ‌های بی‌شمارِ اونها با خِش‌خِشِ روح‌پرورِ خودشون بر من درود می‌فرستادند و انگاری که کلمهٔ عشق را زمزمه می‌کردند. چمنزاران با گل‌های درخشان و معطر در تب و تاب بود. پرنده‌ها در دسته‌های بزرگ در هوا پرواز می‌کردن و بی‌واهمه روی شانه‌ها و بازوانِ من می‌نشستن و با لذت، با بال‌های عزیزِ خودشون به من می‌زدن. و نهایتاً مردمِ این سرزمینِ خوشبخت رو دیدم و شناختم. این اتفاق وقتی افتاد که اونها دور و بر من جمع شدن و من‌و بوسیدن. فرزندانِ خورشید، فرزندانِ خورشیدِ خودشون - آه که چقدر اونا زیبا بودن! هیچ وقت تو زمینِ خودمون، چنین زیبایی‌ای در بنی‌بشر ندیده بودم. شاید فقط تو بچه‌ها، اون هم در سال‌های اولیهٔ عمرشون، بشه بازتابِ ضعیف و کم‌نوری از این زیبایی پیدا کرد. چشمانِ این مردمانِ شاد، با روشناییِ شفافی، می‌درخشید. صورت‌های اون‌ها، با نور منطق و آرامش خاصی که از فهمِ کامل نشأت می‌گرفت، تابناک شده بود، ولی اون صورت‌ها بشّاش بود؛ در کلمات و صدای اونها، اثری از شادیِ کودکانه وجود داشت. آه، از همون لحظهٔ اول، از اولین نگاهی که به اونها انداختم، همه چیزو فهمیدم! این، زمینی بود که ننگِ هُبُوط بر خود نداشت؛ روی آن مردمانی می‌زیستند که گناهی مرتکب نشده بودند. اونها تو چنان بهشتی زندگی می‌کردن که بنا به تمام افسانه‌های بشری، والدینِ اولیهٔ ما، قبل از گناهشون اونجا بودن؛ تنها تفاوت این بود که تمامِ این زمین مثلِ بهشت بود. این مردم که با لذت می‌خندیدند، دور و برِ من حلقه زدن و من‌و در آغوش کشیدن؛ اونها من‌و با خودشون به خونه بردن، و هر کدوم از اونها سعی می‌کرد به من قوتِ قلب بده. آه، اونها از من هیچ سؤالی نپرسیدن، ولی به نظرِ من، بدون اینکه چیزی بِپُرسَن، همه چیزو می‌دونستن و می‌خواستن به سرعت، نشونه‌های رنج رو از صورتِ من پاک کنن.

    \paragraph{}
    و فکر می‌کنین چی شد؟ خب، با اینکه هَمَش یه رویا بود، ولی احساسِ عشقِ اون مردمِ معصوم و زیبا برای همیشه با من خواهد بود، و جوریه که حس می‌کنم هنوز عشقِ اونها از اونجا نثارِ من میشه. من خودم اونها رو دیدم، شناختم و متقاعد شدم؛ من اونها رو دوست داشتم، بعدش به خاطرِ اونها رنج بردم. آه، همون موقع، ناخودآگاه فهمیدم که خیلی از کاراشون‌و اصلاً درک نمی‌کنم؛ به عنوان یه ترقی‌خواهِ متجددِ روسی و یه پترزبورگیِ حقیر، به طرزی کاملاً باورنکردنی فهمیدم که اونها با اینکه این همه چیز می‌دونن ولی هیچ علم و دانشی مشابه ما ندارن. ولی به زودی دریافتم که دانش اونها با ادراک‌هایی کاملاً متفاوت با ما به دست اومده و رشد کرده و اینکه آرزوهای اونها نیز کاملاً متفاوت بود. اونها آرزوی چیزی رو نداشتن و در صلح و صفا زندگی می‌کردن؛ اونها اونجور که ما مشتاق دونِستَنِش هستیم، آرزوی دونستنِ معنیِ زندگی رو نداشتن چرا که زندگی‌شون کامل بود. ولی دانشِ اونها بالاتر و عمیق‌تر از مالِ ما بود؛ چرا که علمِ ما دنبال این می‌گرده که زندگی رو توضیح بده، و آرزو داره معنیِ اون‌و بفهمه تا بتونه عشق‌ورزیدن رو به بقیه یاد بده، در حالی که اونها بدونِ هیچ گونه علمی می‌دونستن که چه جوری زندگی کنن؛ و من این‌و فهمیدم ولی نتونستم دانشِ اونها رو بفهمم. اونها، درختهاشون‌و به من نشون دادن و من نتونستم عشقِ شدیدی رو که باهاش به درختها نگاه می‌کردن، درک کنم؛ جوری بود که انگار با مخلوقاتی مثل خودشون دارن حرف می‌زنن. و اشتباه نخواهد بود اگه بگم که اونها داشتن با درختها گفتگو می‌کردن. بله؛ اونها زبونِ درختها رو پیدا کرده بودن و من مطمئنم که درختها حرفِ اونها رو می‌فهمیدن. اونها به همهٔ طبیعت همین جوری می‌نگریستند - به حیواناتی که با صلح و صفا کنارشون زندگی می‌کردن و به اونها حمله نمی‌کردن، بلکه دوستشون داشتن، با عشق حکومت می‌کردن. اونها به ستاره‌ها اشاره کردن و چیزهایی راجع به اونها گفتن که من نتونستم بفهمم ولی مطمئنم که یه جورایی با ستاره‌ها ارتباط داشتن، نه فقط به طورِ ذهنی بلکه از طریقِ مجرایی زنده و واقعی. آه، این مردم سعی کردن یه کاری کنن که من اونها رو بفهمم، البته اونها همین جوری هم من‌و دوست داشتن، ولی دونستم که هیچ وقت من‌و نخواهند فهمید برای هَمینَم به ندرت دربارهٔ زمینِ خودمون باهاشون حرف می‌زدم. من فقط در حضورِ اونها، زمینی که روش زندگی می‌کردن رو می‌بوسیدم و بی‌سروصدا اونها رو می‌پرستیدم. و اونها متوجه این امر شدن و به من اجازه دادن بدون اینکه از پرستش خودم شرمنده بشم، اونها رو بپرستم، چرا که اونها خودشون خیلی عاشق بودن. اونها از دستِ من ناراحت نمی‌شدن وقتی بعضی موقع‌ها، با اشک، پاهاشون‌و می‌بوسیدم و با لذت منتظرِ عشقی بودم که در جواب، نثار من خواهند کرد. بعضی موقع‌ها با تعجب از خودم می‌پرسیدم چه جوریه که اونها هیچ وقت نمی‌تونن باعث رنجش موجودی مثل من بشن، و هیچ وقت باعث ایجاد حس رشک و حسادت در من نشدن؟ اغلب تعجب می‌کردم که چه جوری می‌تونم اونقدر که چاخان و دروغگو بودم و هیچ وقت از چیزایی که می‌دونستم باهاشون حرف نزدم - چیزایی که البته فکرش هم نمی‌کردن - هیچ وقت وسوسه نشدم که اونها را متحیر کنم یا یه کار مفیدی براشون انجام بدم.

    \paragraph{}
    اونها مثل بچه‌ها سَرکِیف و قبراق بودن. اونها دور و برِ جنگل‌ها و شقایق‌های دوست‌داشتنی‌شون پرسه می‌زدن، آوازهای دلفریب خودشون‌و می‌خوندن؛ غذاهایی که تو جشن‌هاشون می‌خوردن خیلی ساده و سبک بود - میوهٔ درختان، عسلی که از جنگل‌ها به دست میومد و شیری که از حیوانات می‌دوشیدن، حیواناتی که عاشق اونها بودن. زحمتی که برای غذا و لباس می‌کشیدن مختصر بود. اونها عاشق می‌شدن و بچه به دنیا میاوردن، ولی هیچ وقت تو اونها، علامتی از اون شهوانیتِ بیرحم که تقریباً بر تمامِ آدم‌ها غلبه می‌کنه  و سرچشمهٔ تقریباً تمامیِ گناهانِ آدمی بر روی زمینه، ندیدم. اونها موقعِ رسیدنِ بچه‌ها به شدت خوشحال می‌شدن، چون بچه‌ها رو موجوداتِ جدیدی می‌دونستن که می‌خواستن شادی‌شون‌و با اونها قسمت کنن. هیچ گونه دعوا و حسادتی بین اونها نبود و حتی نمی‌دونستن این کلمات چه معنی‌ای میدن. بچه‌هاشون، بچه‌های همه بودن، چرا که همهٔ اونها یه خانوادهٔ واحد رو تشکیل می‌دادن. به ندرت بین اونها بیماری‌ای مشاهده می‌شد، با این حال مرگ وجود داشت؛ ولی افرادِ پیرِشون در نهایتِ آرامش می‌مردن، انگار که می‌خوان بخوابن و به کسایی که دورشون حلقه می‌زدن تا آخرین خداحافظی رو انجام بدن، با لبخندی نورانی و عاشقانه، دعای خیر نثار می‌کردن. من هیچ وقت در چنین مواقعی حزن و اندوه و اشک ندیدم، فقط عشق بود که به آخرین حدِ خلسه و وجد می‌رسید ولی خلسه‌ای آرام، عالی و متفکرانه. می‌شد این جور فکر کرد که اونها بعد از مرگ هم با اون مرحوم در ارتباط بودن و اتحادِ زمینی‌شون با مرگ از بین نمی‌رفت. اونها به سختی حرف من‌و می‌فهمیدن وقتی که ازشون دربارهٔ ابدیت و جاودانگی می‌پرسیدم، ولی به وضوح، بدونِ استدلال، به قدری از این موضوع اطمینان داشتن که انگار چنین سؤالی اصلاً براشون مطرح نبوده. اونها معبدی نداشتن، ولی یه زندگیِ واقعی و حسی لاینقطع از یکتایی در موردِ کلِّ جهان داشتن. اونها مَسلَکی نداشتن، ولی دانشی قطعی داشتن که وقتی لذتِ زمینی اونها به آخرش میرسه، اونوقت برای اونها اَعَم از مرده یا زنده، رضایتی بزرگتر به خاطر تماس با کلِّ جهان بِهِشون دست میده. اونها با اشتیاق منتظر چنین لحظه‌ای بودن، ولی عجله‌ای نداشتن و به خاطرش غصه نمی‌خوردن بلکه به نظر میومد قبلاً در قلبشون مزهٔ اون‌و چشیدن که درباره‌اش با هم صحبت می‌کنن.

    \paragraph{}
    موقع غروب، قبل از اینکه بِرَن بخوابن، به صورت موزون و آهنگین، آوازهایی رو همسرایی می‌کردن. در اون آوازها، تمامِ احساساتی رو که اون روز بِهِشون دست داده بود، بیان می‌کردن، تمامِ شکوهِ اون روز رو به شکل آواز درمیاوردن و دیگه بهش فکر نمی‌کردن. اونها با آواز، طبیعت، دریا و جنگل‌ها رو می‌ستودند. اونها دوست داشتن در مورد همدیگه آواز بسازن و مثل بچه‌ها همدیگه رو تحسین کنن؛ اون آوازها خیلی ساده بودن، ولی از قلبشون تراوش می‌کردن و به قلوب بقیه راه می‌یافتن. و نه تنها در آوازهاشون، بلکه در تمام زندگی‌شون به نظر می‌رسید که کاری جز تحسینِ همدیگه انجام نمی‌دن. به نظر میومد همه عاشق همند که احساسی همه‌شمول و جهانی بود.

    \paragraph{}
    بعضی از آوازهاشون که تشریفاتی و شورانگیز بود برای من اصلاً قابل‌فهم نبود. با اینکه کلمات رو می‌فهمیدم ولی هیچ وقت نتونستم به عمق معنی پی ببرم. این قضیه بدون هیچ تغییری خارج از فهم من باقی موند، ولی با این حال قلب من به طور ناخودآگاه اون‌و بیشتر و بیشتر جذب می‌کرد. من اغلب به اونها می‌گفتم که از مدتها قبل چنین حسی داشته‌ام، که این لذت و شکوه، موقعی که رو زمینِ خودمون بودم به شکل یه اشتیاق مالیخولیایی که بعضی مواقع به اندوهی غیرقابلِ تحمل تبدیل می‌شد، به من دست داده بود، که یه پیش‌آگاهی‌ای از همهٔ اونها و از شکوهِشون در رویاهای قلبم و تصوراتِ مغزم داشته‌ام؛ که اغلب رو زمینِ خودمون نمی‌تونستم بدونِ ریختنِ اشک به غروبِ خورشید نگاه کنم... که در کینهٔ من نسبت به آدم‌های روی زمین همیشه یه دلتنگیِ کُشَندِه وجود داشت: چرا برای اینکه اَزَشون متنفر نباشم مجبور بودم دوستشون نداشته باشم؟  چرا نمی‌تونستم اونها رو ببخشم؟ و در عشقِ من نسبت به اونها، یه غمِ کُشَندِه وجود داشت: چرا برای اینکه اونها را دوست نداشته باشم، مجبور بودم ازشون متنفر نباشم؟ اونها به من گوش می‌دادن، و می‌دیدم که نمی‌تونن چیزی رو که میگم بفهمن، ولی از اینکه این چیزها رو بهشون گفتم پشیمون نشدم: می‌دونستم که شدتِ دلتنگی کشندهٔ من‌و نسبت به کسایی که ترکشون کرده بودم درک می‌کنن. ولی وقتی اونها با چشم‌های مهربونِ مملو از عشقشون به من نگاه می‌کردن، وقتی احساس می‌کردم که در حضور اونها، قلبِ من هم به اندازهٔ قلبِ اونها معصوم میشه، احساسِ کاملِ زندگی نفسم‌و بند می‌آورد و در سکوت، اونها را می‌پرستیدم.

    \paragraph{}
    آه، الآن همه به من می‌خندند، و میگن که کسی نمی تونه با چنین جزئیاتی که من دارم میگم، خواب ببینه، که من فقط یه رویا دیدم و یا اینها فقط توهماتیه که تو هذیان بِهِم دست داده و وقتی بیدار شدم، این جزئیات رو از خودم درآوردم. و وقتی بهشون گفتم که شاید همینطور که می‌گین باشه، اوه خدایا، نمی‌دونین چه جوری با صدای بلند بِهِم خندیدن و من باعث چه خوشحالی‌ای شدم! آه، بله، البته من مغلوبِ احساسِ توی خوابم شدم و این تنها چیزی بود که در قلبِ به‌شدت‌زخم‌خوردهٔ من باقی موند. ولی شکل‌ها و تصاویرِ واقعیِ رویای من، یعنی چیزهایی که واقعاً موقع خوابم دیدم، با چنان ‌هارمونی‌ای آمیخته بود، چنان دوست‌داشتنی و دلربا و چنان واقعی بود که وقتی بیدار شدم، البته، قادر نبودم اونها رو با زَبونِ ضعیفِمون توصیف کنم، جوری که اونها تو ذهنِ من به صورتِ تصویرِ محوی متبلور شدن؛ و شاید من واقعاً بعدش مجبور شدم جزئیات رو از خودم دربیارم، و بنابراین به میلِ احساساتِ خودم، اونها رو کمی دستکاری کنم تا حداقل بتونم هر چی زودتر بخشی از اونها رو به شما انتقال بدم. ولی از طرفِ دیگه، نمی‌تونم در باور کردنش به شما کمکی بکنم. شاید هزاران بار روشن‌تر، شادتر و لذت‌بخش‌تر از اون چیزی بود که من دارم میگم. با اینکه من اینها رو تو رویا دیدم، ولی باید واقعی بوده باشن. می‌دونین، من به شما رازی رو خواهم گفت؛ شاید اصلاً خواب نبود! چرا که بعدش چیزهایی اتفاق افتاد که به قدری مهیب و به قدری واقعی بود که نمی‌شد گفت تصوراتِ توی رویا بودن. مُمکِنِه قلبِ من سرچشمهٔ رویا بوده باشه ولی آیا میشه قلبِ من به تنهایی قادر باشه حوادث مهیبی رو که بعدش برای من اتفاق افتاد به وجود آورده باشه؟ چه جوری من می‌تونستم به تنهایی اینها رو بسازم یا تو خوابم تصور کنم؟ آیا قلبِ حقیر و مغزِ دمدمی‌مزاج و مبتذلِ من می‌تونستن به چنان سطحی از دریافتِ حقیقت نائل بشن؟ آه، خودتون قضاوت کنین: تابحال من این‌و پنهان می‌کردم، ولی الآن می‌خوام حقیقت‌و بگم. واقعیت اینه که من... تمام اونها رو فاسد کردم!

    \paragraph{}
    بله، بله، آخرش قضیه به این ختم شد که من همهٔ اونها رو فاسد کردم! چی شد که این جوری شد، نمی‌دونم، ولی کلِّ ماجرا رو به وضوح به خاطر میارم. رویای من هزاران سال طول کشید و فقط یه حس کُلّی در من باقی گذاشت. فقط این‌و می‌دونم که من مسبب گناه و هبوط اونها بودم. مثل یه انگلِ بی‌خاصیت، مثل میکروبِ طاعونی که کلِّ مملکت رو آلوده می‌کنه، من هم تمام اون زمین‌و به گَند کشیدم، زمینی که قبل از اومدنِ من، اونقدر شاد و بی‌گناه بود. اونها یاد گرفتن دروغ بگن، به دروغِ خودشون افتخار کنن، و با افسونِ دروغگویی آشنا شدن. آه، اولش شاید همینطوری ناخودآگاه اتفاق افتاد، با یه شوخی، کرشمه، با نقش‌بازی‌کردنِ عاشقانه، شاید واقعاً با یه میکروب؛ ولی اون میکروبِ ناراستی به درونِ قلب‌ها راه پیدا کرد و باعثِ شادی اونها شد. بَعدِش خیلی زود، شهوانیت به وجود اومد، شهوانیت، حسادت رو به وجود آورد، حسادت، ظلم رو و همین طور الی آخر...

    \paragraph{}
    آه، من نمی‌دونم، یادم نمیاد؛ ولی به زودی، خیلی زود، اولین خون به زمین ریخته شد. اونها گیج شدن و ترسیدن، و شروع کردن به جدا شدن و دسته‌دسته شدن. اونها اجتماعاتی رو تشکیل دادن ولی ضد هم‌دیگه. سرزنش‌ها و انتقادها شروع شد. اونها با شرم، آشنا شدن و شرم باعثِ شد به تقوا رو بیارن. مفهوم شرافت قدم به میدان گذاشت و هر گروهی شروع کرد به تکون دادنِ پرچمِ خودش. اونها حیوون‌ها رو شکنجه می‌کردن، و حیوون‌ها از دست اونها به جنگل‌ها پناه بردن و دشمنشون شدن. اونها برای جدا شدن، برای انزوا، برای فَردیَت، برای مالِ من و مالِ تو، دست و پا می‌زدن. اونها شروع کردن به زبون‌های مختلف صحبت کردن. اونها غصه خوردن و عاشقِ‌غصه‌خوردن‌بودن رو یاد گرفتن؛ اونها تشنهٔ رنج بردن بودن و می‌گفتن حقیقت فقط از طریق رنج بردن به دست میاد. سپس دانش به وجود اومد. به محضِ اینکه بدجنس و شرور شدن، شروع کردن به صحبت دربارهٔ برادری و همنوع‌دوستی، و این افکار رو درک کردن. به محض اینکه جنایتکار شدن، عدالت رو اختراع کردن و برای نظارت بر عدالت، تمامِ قوانینِ حقوقی رو تنظیم کردن و برای اطمینان از اجرای عدالت، گیوتین رو ساختن. اصلاً به خاطر نمی‌آوردن که چی‌و از دست دادن، در واقع انکار می‌کردن که یه زمانی شاد و معصوم بودن. حتی به تصورِ این که قبلاً شاد بودن می‌خندیدن، و می‌گفتن اون فقط یه خواب و خیال بوده. حتی نمی‌تونستن به درستی، یه همچین چیزی‌و تصور کنن. با اینکه تمامِ ایمانِ خودشون به شادیِ گذشته رو از دست داده بودن و می‌گفتن که هَمَش یه افسانه‌س، به قدری می‌خواستن یه بار دیگه شاد و معصوم باشن که مثلِ بچه‌ها در برابر این آرزو تسلیم شده و اَزَش یه بُت ساختن، معبدها ساختن و ایدهٔ خودشون‌و و آرزوی خودشون‌و می‌پرستیدن؛ با اینکه کاملاً اعتقاد داشتن که اون گذشته، غیرقابل‌دسترسیه و تحقق نمی‌پذیره ولی با این حال، در برابرش سرِ تعظیم فرود می‌آوردن و با اشک، اون‌و می‌ستودن! به هر حال، اگه می‌شد که به شرایطِ معصومیت و شادی‌ای که از دست داده بودن برگردن و اگه کسی دوباره اون‌و بِهِشون نشون می‌داد و اَزَشون می‌پرسید که آیا می‌خواین برگردین، به طورِ قطع قبول نمی‌کردن. اونها به من جواب دادن: «ممکنه ما متقلب، نابکار و نادرست باشیم، این‌و می‌دونیم و از این بابت اشک می‌ریزیم و ناراحتیم؛ ما شاید خودمون‌و بیشتر از اون قاضیِ رحیمی که در موردِ کارهامون داوری خواهد کرد و ما اسمشو نمی‌دونیم، شکنجه و تنبیه می‌کنیم. ولی ما دانش داریم و از طریق اون، حقیقت رو خواهیم یافت و آگاهانه بِهِش می‌رسیم. دانش برتر از احساساته؛ آگاهی به زندگی برتر از خود زندگیه. دانش به ما خِرَد میده، خِرَد، قوانین رو به وجود میاره، و علمِ به قوانینِ شادی، برتر از خود شادیه».

    \paragraph{}
    این، چیزی بود که اونها گفتن، و بعد از گفتنِ چنین چیزهایی، همه، خودشون‌و بیشتر از هر کسِ دیگه‌ای دوست داشتن، و در واقع جورِ دیگه‌ای نمی‌تونستن باشن. هَمَشون به قدری نسبت به حقوقِ شخصیتیِ خودشون حساس شده بودن که نهایتِ سعی‌شون رو برای کاهش و تخریب اون حقوق در بقیه به کار می‌بردن، و اون‌و به عنوان مهمترین چیزِ زندگی‌شون قلمداد می‌کردن. بَعدِش، بَردِگی به وجود اومد، حتی بَردِگی اختیاری؛ ضعیف، مشتاقانه خودش‌و تسلیمِ قوی می‌کرد، به شرطی که قویه کمکش کنه تا بتونه ضعیف‌تر از خودش‌و به اطاعت وادار کنه. بَعدِش، قدیسانی پیدا شدن که بینِ مردم می‌رفتن، اشک می‌ریختن و در موردِ عزت نفس‌شون، در مورد فقدان هماهنگی و تناسب بین‌شون و در مورد بی‌شرمی اونها، باهاشون صحبت می‌کردن. مردم به اون قدیس‌ها می‌خندیدن یا سنگسارشون می‌کردن. خون مقدس در آستانِ معابد ریخته شد. سپس مردمانی به پاخاستند که به این می‌اندیشیدن که چه جوری دوباره همهٔ آدمها رو دورِ هم جمع کنیم، جوری که هر کس با این که خودش‌و بیشتر از بقیه دوست داره، تو کارِ بقیه دخالت نکنه و همه تو یه جامعهٔ نسبتاً هماهنگ زندگی کنن. جنگ‌های منظمی به خاطر این عقیده درگرفت. با این حال، همهٔ رزمندگان، اعتقادِ راسخ داشتن که دانش، خِرَد و غریزهٔ حفظِ جانِ خود، باعث میشه که آدمها سرآخر با هم متحد شده و جوامعِ هماهنگ و منطقی‌ای رو تشکیل بدن؛ و بنابراین، ضمناً، برای تسریعِ امور، «خِرَدمَند» سعی می‌کرد به سرعتِ هر چه تمام‌تر، همهٔ کسانی رو که «بی‌خرد» بودن و متوجه افکارِ اون نمی‌شدن، از بین بِبَرِه، چرا که ممکن بود «بی‌خرد»، پیروزی و موفقیت اون‌و به تأخیر بندازه. ولی غریزهٔ حفظِ جانِ خود، به شدت ضعیف شد؛ مَردانی متکبر و هوسران پیدا شدن که یا همه یا هیچ رو می‌خواستن. برای بدست آوردن همه چیز، اونها متوسل به جنایت می‌شدن و اگه موفق نمی‌شدن، خودکشی می‌کردن. مذاهبی پیدا شدن که تفکرشون عدمِ وجود و تخریبِ خود به خاطرِ صلحِ ابدی بود. سرآخر، این مردم از رنج بی‌معنیِ خود به ستوه اومدن و علائمِ رنج تو صورتهاشون پیدا شد و بَعدِش ادعا کردن که رنج بردن، زیباست، چرا که فقط رنج بردن معنی داره. اونها، رنج رو در آوازهاشون ستایش می‌کردن. من بینِ اونها می‌رفتم، دستهام‌و مشت می‌کردم و به حالشون اشک می‌ریختم، ولی شاید بیشتر از گذشته، وقتی که هیچ رنجی در صورتشون نبود و وقتی که معصوم و بسیار دوست‌داشتنی بودن، دوستشون داشتم. من اون زمینی رو که اونها آلوده کرده بودن حتی بیشتر از وقتی که مثلِ بهشت بود، دوست داشتم، شاید فقط به خاطرِ این که غصه روی اون به وجود اومده بود. افسوس! من همیشه عاشق غصه و رنج بودم، ولی فقط برای خودم، برای خودم؛ ولی من به خاطر اونها گریه می‌کردم و دلم به حالشون می‌سوخت. من با ناامیدی، دستهام‌و به طرفشون دراز می‌کردم و خودم‌و سرزنش می‌کردم، خودم‌و لعنت و تحقیر می‌کردم. من به اونها گفتم که همهٔ اینها تقصیر مَنه، فقط من؛ چرا که من باعث فساد، ناپاکی و ناراستیِ اونها شدم. من بِهِشون التماس می‌کردم که من‌و مصلوب کنن، من بِهِشون یاد دادم که چه جوری صلیب بسازن. من نمی‌تونستم خودم‌و بُکُشَم، قدرتش‌و نداشتم، ولی می‌خواستم توسطِ اونها رنج بکشم. من برای رنج بردن، بی‌تابی می‌کردم، آرزو می‌کردم که خونِ من تا آخرین قطره، با عذاب از تَنَم بیرون بِرِه. ولی اونها فقط به من خندیدن، و سرآخر به من به چشمِ یه دیوانه نگاه می‌کردن. اونها به من حق دادن، گفتن که فقط چیزی رو به دست آوردن که خودشون می‌خواستن، و تمام چیزی که الان وجود داره نمی‌تونست جورِ دیگه‌ای باشه. آخرش به من گفتن که دارم خطرناک میشم و اگه جلوی زبونم‌و نگیرم، میندازَنَم تو دیوونه‌خونه. بَعدِش چنان اندوهی وجودم‌و فراگرفت که قلبم فشرده شد، و احساس کردم که دارم می‌میرم؛ و بَعدِش... بَعدِش بیدار شدم.

    \paragraph{}
    صبح بود، یعنی هنوز آفتاب نزده بود، ولی حدود ساعت شش بود. من تو همون صندلیِ دسته‌دار بیدار شدم؛ شَمعَم کاملاً آب شده بود؛ تو اتاقِ سروان، همه خوابیده بودن، و همه جا سکوت حکمفرما بود، چیزی که تو آپارتمانِ ما سابقه نداشت. اوّلِش با تعجب از جام پریدم: همچین چیزی قبلاً برام اتفاق نیفتاده بود حتی در بی‌اهمیت‌ترین جزئیات؛ مثلاً من هیچ وقت این جوری تو صندلی دسته‌دارم به خواب نرفته بودم. در حالی که ایستاده بودم و به خودم میومدم، ناگهان چشمم به طپانچه‌ام افتاد که آماده و پُر شده بود - ولی ناگهان به گوشه‌ای پرتش کردم! آه، حالا، زندگی، زندگی! من دستهام‌و بالا بردم و حقیقتِ ابدی رو صدا زدم، نه با کلمات بلکه با اشک. بله، زندگی و خبرهای خوب! آه، در اون لحظه من تصمیم گرفتم تا خبرهای خوبی به همه بدم و این تصمیم رو البته برای تمام زندگی‌ام گرفتم. من رفتم که این خبر رو پخش کنم، من می‌خواستم که خبری رو پخش کنم - ولی خبرِ چی رو؟ خبر حقیقت، چیزی که دیدمش، چیزی که با چشم‌های خودم دیدمش، چیزی که با تمامِ شکوه و عظمتش دیدم.

    \paragraph{}
    و از اون موقع تا حالا من دارم موعظه می‌کنم! علاوه بر این، من همهٔ اونهایی رو که به من می‌خندن بیشتر از بقیه، دوست دارم. اینکه چرا این جوریه، من نه می‌دونم و نه می‌تونم توضیح بدم، ولی بی‌خیالش، بِذار همین جوری بِمونه. به من می‌گن که قاطی کردم و دارم چرت و پرت می‌گم، و اگه الان من قاطی کردم و چرت و پرت می‌گم، بعداً چی میشم؟ در واقع حرفشون درسته: من قاطی کردم و چرت و پرت میگم و شاید با گذشت زمان بدتر هم بشم. و البته قبل از اینکه موعظه کردن‌و یاد بگیرم، یعنی بفهم چه کلماتی باید بگم و چه کارهایی باید بکنم، ممکنه سوتی‌های زیادی بدم، چرا که کارِ بسیار سختیه. هَمَش مثل روز برام روشنه، ولی ببینین، کیه که اشتباه نمی‌کنه؟ و با این حال، می‌دونین، همه دارن برای یه هدفِ مشابه، اشتباه می‌کنن، همه، از آدمِ عاقل گرفته تا پَست‌ترین راهزن در همون مسیر کوشش می‌کنن، فقط راه‌هاشون با هم فرق داره. این یه حقیقتِ قدیمیه ولی چیزی که جدیده، اینه: من نمی‌تونم زیاد اشتباه کنم. برای اینکه حقیقت رو فهمیدم؛ من فهمیدم و می‌دونم که مردم می‌تونن زیبا و شاد باشن بدون اینکه قدرتِ زندگی روی زمین‌و از دست بِدَن. من باور نمی‌کنم و نمی‌تونم باور کنم که شرارت، حالتِ عادیِ بنی‌بَشَرِه. و اونها فقط به همین عقیدهٔ من می‌خندن. ولی چه جوری می‌تونم تو باورکردنش کمک کنم؟ من حقیقت رو فهمیده‌ام - نه اینکه از خودم ساخته باشم، بلکه دیدمش، دیدمش و تصویر زندهٔ اون، روحم‌و برای همیشه پر کرده. اون‌و با چنان کیفیت و کمالی دیدم که نمی‌تونم باور کنم دستیابی بِهِش برای آدمها غیرممکنه. و بنابراین چطور ممکنه که اشتباه کنم؟ شکی نیست که ممکنه لغزش‌هایی داشته باشم، و شاید با یه زبونِ دست‌دوم صحبت کنم ولی این امر مدت زیادی طول نخواهد کشید: تصویر زندهٔ چیزی که دیدم، همیشه با من خواهد بود و همیشه من‌و اصلاح و راهنمایی خواهد کرد. آه، من پر از شهامت و طراوتم، و به راهم ادامه خواهم داد حتی اگه هزار سال طول بکشه! می‌دونین، اَوَلِش می‌خواستم این واقعیت‌و که من اونها رو فاسد کردم، پنهان کنم، ولی این کار، اشتباه بود - اولین اشتباهم بود! ولی حقیقت در گوشم نجوا کرد که دارم دروغ می‌گم، و من‌و از خطا حفظ کرد و اصلاحم کرد. ولی چه جوری بهشت بسازیم - نمی‌دونم، چون نمی‌دونم چه جوری در قالب کلمات بیانِش کنم. بعد از رویام، کنترلِ کلمات، همهٔ کلماتِ مهم و ضروی، رو از دست داده‌ام. ولی مهم نیست، جلو میرم و به صحبت کردن ادامه خواهم داد، و تسلیم نخواهم شد، چرا که به هر حال با چشم‌های خودم اون‌و دیدم هر چند که نمی‌تونم چیزی رو که دیدم بیان کنم. ولی مسخره‌کنندگان، این‌و نمی‌فهمن. اونها میگن هَمَش یه رویا بوده، یه هذیون، یه توهم. آه! شاید همینطور باشه! و اونها چقدر سربلند و مغرورند! یه رویا! ولی رویا چیه؟ و آیا زندگیِ ما، یه رویا نیست؟ من باز هم خواهم گفت. فرض کنین که این بهشت، هیچ وقت اتفاق نیفته (به نظر من)، ولی با این حال من به موعظه کردن در موردِ اون ادامه میدم. و چقدر ساده‌س: یه روز، تو یه ساعت، همه چیز یه دفعه مرتب میشه! مهمترین چیز، دوست‌داشتنِ بقیه به اندازهٔ خودتونه، این مهمترین چیز و همه چیزه: به چیزِ دیگه‌ای نیاز نیست - تو یه لحظه می‌فهمین که چی کار باید بکنین. و با این حال، این یه حقیقتِ قدیمیه که یه میلیون بار گفته شده و باز هم گفته میشه - ولی هنوز جزئی از زندگی‌مون نشده! آگاهی نسبت به زندگی، برتر از خود زندگیه، علم به قوانینِ شادی از خودِ شادی بَرتَرِه - این چیزیه که باید باهاش مبارزه کرد. و من این کارو میکنم. اگه فقط همه بخوان، میشه تو یه لحظه، همه چیزو درست کرد.

    \paragraph{}
    و من به دنبالِ اون دختر کوچولو رفتم... و خواهم رفت و خواهم رفت...

\end{document}