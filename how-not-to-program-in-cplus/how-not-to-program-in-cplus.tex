% Default Compiler: txs:///xelatex
% Default Bibliography Tool: BibTex

\documentclass[10pt, a4paper]{book}
\usepackage[table, svgnames, usenames,dvipsnames]{xcolor} \usepackage{array}
\usepackage{titlesec}
\usepackage[linktocpage=true,colorlinks,citecolor=blue,pagebackref=true]{hyperref}
\usepackage[top=30mm, bottom=30mm, left=30mm, right=30mm]{geometry}
\usepackage[utf8]{inputenc}
\usepackage{tcolorbox}
\usepackage{listings,lstautogobble}
\usepackage{epsfig,graphicx,subfigure,amsthm,amsmath}
\usepackage{caption}
\usepackage{float}
\usepackage{hyphenat}
\usepackage{setspace}
\usepackage{pgffor}
\makeatletter

%\onehalfspacing
\doublespacing

\usepackage{xepersian}
\settextfont[Scale=1]{IRNazanin}
\defpersianfont\mfo[Scale=1]{IRNazanin}
\setlatintextfont[Scale=0.8]{Doulos SIL}

\renewcommand{\chaptername}{فصل}

\titleformat
{\chapter} % command
[display] % shape
{\bfseries\Large\itshape} % format
{فصل \ \thechapter} % label
{0.5ex} % sep
{
\rule{\textwidth}{1pt}
\vspace{1ex}
\centering
} % before-code
[
\vspace{-0.5ex}%
\rule{\textwidth}{0.3pt}
] % after-code

\DefaultMathsDigits

\definecolor{customblue}{RGB}{235,241,245}
\definecolor{light-gray}{gray}{0.95}
\lstdefinestyle{C++Style}{%
backgroundcolor=\color{customblue},
breaklines=true,
basicstyle=\scriptsize\ttfamily,
keywordstyle=\color{blue},
%commentstyle=\color{OliveGreen}\textit,
commentstyle=\color{OliveGreen},
stringstyle=\color{red},
numbers=left,
numberstyle={\tiny\lr},
showspaces = false,
showstringspaces = false,
tabsize = 2,
frame=single,
xleftmargin=5pt,
xrightmargin=3pt,
language = C++,
aboveskip = 20pt,
rulecolor=\color{black},
captiondirection=RTL,
autogobble=true
}

\lstnewenvironment{C++Code}[1][]
{%
\lstset{style=C++Style, #1}%
}{%
}

\begin{document}
    \title{چگونه در C++ برنامه ننویسیم\\
    یا\\
    چرا 5986 = 2 + 2}
    \author{استیو اُلین\\
    ترجمهٔ وحید مواجی\\
    }
    \date{آذر ۱۳۸۴}
    \frontmatter                            % only in book class (roman page #s)
    \maketitle                              % Print title page.
    \tableofcontents                        % Print table of contents
    \mainmatter

    \part{برنامه‌ها}

\titleformat{\section}{\bfseries}{برنامه \thesection, }{0em}{}
\setcounter{section}{0}
\renewcommand*\thesection{\arabic{section}}

\section*{مقدمه}
رنج و زحمت، ابزاری شگفت برای یادگیری می‌باشد. طبیعت از طریقِ رنج می‌گوید: «این کار را انجام نده!»؛ اگر شما یک برنامه‌نویس باشید، سهمِ خود از رنج را برده‌اید. این امر معمولاً حدود 2 نیمه‌شب اتفاق می‌افتد وقتی آخرین باگی را که برای دو هفته شما را شکنجه می‌داد، می‌یابید.

این کتاب پر از برنامه‌های باگ‌دار است و به شما اجازه می‌دهد که از بدشانسی‌های بقیهٔ برنامه‌نویس‌ها عبرت بگیرید. این شاملِ باگ‌هایی می‌شود که من یافته‌ام، باگ‌هایی که دوستانم یافته‌اند و باگ‌هایی که بقیهٔ برنامه‌نویسان یافته‌اند. هر برنامه یک تجربه برای یادگیری است.

برنامه‌های ارائه شده در این جا، طوری طراحی شده‌اند که تا آنجا که ممکن است به برنامه‌های واقعی شبیه باشند. هر برنامه، یک کارِ ساده را انجام می‌دهد یا از یکی از ویژگی‌های زبان \lr{\texttt{C++}}، استفاده می‌نماید. خبرِ بد این است که این برنامه‌ها کار نمی‌کنند. خبرِ خوب این است که همهٔ آنها نسبتاً کوچک هستند و لازم نیست که شما مثلاً یک برنامه 750000 خطی را بالا و پایین کنید تا مشکل را بیابید.

برخی افراد معتقند که با تکنولوژیِ جدیدِ کامپایلرها، اکثرِ این خطاها یافته می‌شوند. متأسفانه، خطاهای بسیاری وجود دارد که کامپایلرها نمی‌توانند آنها را بیابند.

برای مثال،از \lr{Spell Checker}ها انتظار می‌رود که خطاهای املایی را حذف کنند. ولی می‌توانید در این جملهٔ کاملاً بی‌معنی خطای املایی پیدا کنید؟ «خروس‌های بویناک یا یک الهه فکر می‌کنند چون در طرفِ دیگر، این قالب ممکن است سوختی از پیکان‌ها باشد!» (یک \lr{Spell Checker} نمی‌تواند هیچ خطای املایی در این جمله پیدا کند).

بنابراین سعی کنید خطاها را بیاید. اگر به مشکل برخوردید، ما راهنمایی‌هایی را فراهم کرده‌ایم که به شما کمک شود. همچنین پاسخ‌ها در انتهای کتاب می‌باشند. این امر در تضاد با زندگیِ واقعی است که هیچ راهنمایی در آن وجود ندارد و جوابی در انتهای کتابی برای آن نمی‌یابید.

این کتاب تقدیم می‌شود به همهٔ برنامه‌نویسانی که روزهای متمادی با برنامه‌های پیچیده، باگ‌دار و پر از مشکل دست و پنجه نرم می‌کنند و مجبورند که رازِ معمای آنها را بگشایند.


\newcommand{\@chaptersspath}{parts/part1/chapters}

\foreach \n in {1,2,...,11}{
    \IfFileExists{\@chaptersspath/chapter\n/chapter\n}{
        \input{\@chaptersspath/chapter\n/chapter\n}
    }
}

    \part{راهنمایی‌ها}
\addtocontents{toc}{\protect\setcounter{tocdepth}{0}}

\titleformat{\section}{\bfseries}{راهنمایی \thesection}{0em}{}
\setcounter{section}{0}
\renewcommand*\thesection{\arabic{section}}

\newcommand{\@hintspath}{parts/part2/hints/hint}

\foreach \n in {1,2,...,361}{
    \IfFileExists{\@hintspath\n}{
        \input{\@hintspath\n}
    }
}

    \part{جواب‌ها}
\addtocontents{toc}{\protect\setcounter{tocdepth}{0}}

\titleformat{\section}{\bfseries}{جواب \thesection}{0em}{}
\setcounter{section}{0}
\renewcommand*\thesection{\arabic{section}}

\newcommand{\@answerspath}{parts/part3/answers/answer}

\foreach \n in {1,2,...,115}{
    \IfFileExists{\@answerspath\n}{
        \input{\@answerspath\n}
    }
}
\end{document}