\section{}
\paragraph{}\label{answer:59}
مشکل این جاست که ما یک تابع سازنده کپی تعریف نکرده‌ایم. وقتی این اتفاق می‌افتد، \lr{\texttt{C++}} یکی برای شما تعریف می‌کند و معمولاً این کار را خوب انجام نمی‌دهد. تابع سازنده کپی این گونه تعریف شده است:
\begin{LTR}
    %@formatter:off
    \begin{lstlisting}[style=C++Style]
        var_array(const var_array &other) {
            data = other.data;
            size = other.size;
        }
    \end{lstlisting}
    %@formatter:on
\end{LTR}

تابع سازنده کپی، برای ساختن یک کپی از \lr{\texttt{an\_array}} برای تابع \lr{\texttt{store\_it}} فراخوانی می‌شود. اشاره‌گر به داده، کپی می‌شود. وقتی \lr{\texttt{var\_array::\textasciitilde var\_array}} در انتهای \lr{\texttt{pushy}} فراخوانی می‌شود، داده را به \lr{\texttt{heap}} برمی‌گرداند. وقتی در انتهای \lr{\texttt{main}}، \lr{\texttt{var\_array::\textasciitilde var\_array}} فراخوانی می‌شود، همان داده را به \lr{\texttt{heap}} برمی‌گرداند. از آن جا که یک مکان حافظه را دو بار پاک می‌کنیم، نتیجه این می‌شود که یک \lr{\texttt{heap}} خراب داریم.

همیشه به نحوی، تابع سازنده کپی را تعریف کنید. سه راهِ عمدهٔ این کار این می‌باشد:
\begin{enumerate}
    \item به طور ضمنی آن را تعریف کنید.
    \item اگر می‌خواهید که هیچ کس قادر به فراخوانی آن نباشد، آن را بصورت \lr{\texttt{private}} تعریف کنید:
    \begin{LTR}
        %@formatter:off
        \begin{lstlisting}[style=C++Style]
            var_array (const var_array &);
            // No one can copy var_arrays
        \end{lstlisting}
        %@formatter:on
    \end{LTR}

    \item اگر از پیش‌فرض استفاده می‌کنید، آن را در توضیح بیاورید:
    \begin{LTR}
        %@formatter:off
        \begin{lstlisting}[style=C++Style]
            // Copy Constructor defaults
        \end{lstlisting}
        %@formatter:on
    \end{LTR}

\end{enumerate}

بدین ترتیب، به بقیه می‌گویید که برنامهٔ شما را چطور بخوانند و لذا پیش‌فرض \lr{\texttt{C++}} مشکل‌زا نخواهد بود.