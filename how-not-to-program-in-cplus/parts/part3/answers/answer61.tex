\section{}
\paragraph{}\label{answer:61}
مشکل این جاست که عبارت:
\begin{LTR}
    %@formatter:off
        \begin{lstlisting}[style=C++Style]
            5 const char *volatile in_port_ptr =
            6 (char *)0xFFFFFFE0;
        \end{lstlisting}
    %@formatter:on
\end{LTR}

به \lr{\texttt{C++}} می‌گوید که اشاره‌گر از نوع \lr{\texttt{volatile}} است. داده‌ای که به آن اشاره می‌شود از نوع \lr{\texttt{volatile}} نیست. نتیجه این است که بهینه‌ساز، عدمِ وجود را برای ما بهینه‌سازی می‌کند. راه‌حل این است که \lr{\texttt{volatile}} را جایی قرار دهیم که دادهٔ مورد اشاره را دستکاری می‌کند. ما همچنین یک \lr{\texttt{const}} به اعلان خود اضافه کرده‌ایم تا مطمئن شویم که اشاره‌گر قابل تغییر نیست. در نتیجه، به اعلان‌های زیر می‌رسیم:
\begin{LTR}
    %@formatter:off
        \begin{lstlisting}[style=C++Style]
            4 // Input register
            5 volatile char *const in_port_ptr =
            6 	(char *)0xFFFFFFE0;
            7
            8 // Output register
            10 volatile char *const out_port_ptr =
            11 		(char *)0xFFFFFFE1;
        \end{lstlisting}
    %@formatter:on
\end{LTR}

این به \lr{\texttt{C++}} می‌گوید که:
\begin{itemize}
    \item \lr{\texttt{in\_port\_ptr}} یک اشاره‌گر ثابت است و قابل تغییر نیست.
    \item \lr{\texttt{*int\_port\_ptr}} از نوع \lr{\texttt{volatile char}} است که مقدار آن را می‌توان وَرای قوانین عادی برنامه‌نویسی \lr{\texttt{C++}} تغییر داد.
\end{itemize}