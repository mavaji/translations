\section{}
\paragraph{}\label{answer:43}
در \lr{\texttt{MS-DOS}} چیزی مانند این را دریافت می‌کنید:
\LTR\noindent
\lr{\texttt{The answer is 4C:>\#\\
(\# is the cursor)}}
\RTL
در یونیکس چیزی مثل این:
\LTR\noindent
\lr{\texttt{The answer is 4\$ \#}}
\RTL
مسأله این است که برنامه‌نویس در انتهای عبارت \lr{\texttt{std::cout}}، یک \lr{\texttt{end of line}} قرار نداد. نتیجه این است که برنامه اجرا می‌شود، چیزی چاپ می‌کند و پایان می‌یابد در حالی که مکان‌نما را در انتهای خط باقی‌می گذارد. سپس پردازندهٔ فرمان اجرا شده و اعلان خود را (\lr{\texttt{C:>}} برای \lr{\texttt{MS-DOS}} و \lr{\texttt{\$}} برای یونیکس) درست بعد از خروجی برنامه قرار می‌دهد.

چیزی که برنامه‌نویس باید می‌نوشت این است:
\begin{LTR}
    %@formatter:off
    \begin{lstlisting}[style=C++Style]
        std::cout << "The answer is " << result << '\n';
    \end{lstlisting}
    %@formatter:on
\end{LTR}
