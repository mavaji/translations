\section{}
\paragraph{}\label{answer:108}
فراخوانی \lr{\texttt{printf}}، هر رشته‌ای را که به آن بدهید، چاپ می‌کند. اگر به یک رشته کاراکتری، 1 اضافه کنید، رشته را با حذف اولین کاراکتر خواهید داشت. بنابراین:
\LTR\noindent
\lr{\texttt{printf("-xxx") prints -xxx\\
printf("-xxx" + 1) prints xxx
}}
\RTL
عبارت \lr{\texttt{((flags \& 0x4) != 0)}} بسته به این که بیت دارای مقدار باشد یا نه، 0 یا 1 برمی‌گرداند. اگر بیت دارای مقدار باشد، برنامه‌نویس، \lr{\texttt{-word}} را چاپ می‌کند \lr{\texttt{("-word" + 0)}}. اگر بیت، مقدار نداشته باشد، خروجی برابر \lr{\texttt{word}} خواهد بود \lr{\texttt{("-word" + 1)}}.