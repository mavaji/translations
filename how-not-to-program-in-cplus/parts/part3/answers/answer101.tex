\section{}
\paragraph{}\label{answer:101}
در تابع مخربِ کلاسِ پایه، تابع \lr{\texttt{clear}} را فراخوانی می‌کنیم. این تابع، تابع مجازی محضِ \lr{\texttt{delete\_data}} را فراخوانی می‌کند. در طول عملِ تخریب، کلاس مشتق ابتدا از بین می‌رود. وقتی کلاس مشتق از بین رفت، تعریف \lr{\texttt{delete\_data}} هم از بین می‌رود. سپس تابع مخرب کلاس پایه فراخوانی می‌شود. در این حالت، کلاس لیستِ ما، به طور غیرمستقیم، \lr{\texttt{delete\_data}} را فراخوانی می‌کند که مجازی محض است. از آنجا که هیچ کلاس مشتقی وجود ندارد، سیستمِ زمان‌اجرا برنامه را متوقف می‌کند.

در توابع سازنده یا مخرب یک کلاسِ مجرد، توابع مجازی محض را فراخوانی نکنید.