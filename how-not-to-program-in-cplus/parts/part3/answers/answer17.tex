\section{}
\paragraph{}\label{answer:17}
مشکل این است که برنامه‌نویس از «و» بیتی \lr{\texttt{(\&)}} به جای «و» منطقی \lr{\texttt{(\&\&)}} استفاده کرده است. «و» بیتی دو عدد به ما می‌دهد:
\LTR\noindent
\lr{\texttt{|~~~3~0011}}\\
\lr{\texttt{|\&~12~1100}}\\
\lr{\texttt{|=========}}\\
\lr{\texttt{|~~~0~0000}}

\RTL
لذا اگر نتیجه صفر است، عبارت \lr{\texttt{if}} اجرا نشده و قسمت \lr{\texttt{else}} اجرا می‌شود. بعضی برنامه‌نویسان از مختصرنویسی \lr{\texttt{if(x)}} برای \lr{\texttt{if(x != 0)}} استفاده می‌کنند. این مثالی است برای این که چرا من از مختصرنویسی خوشم نمی‌آید. راه بهترِ نوشتنِ عبارت \lr{\texttt{if}} این است:
\begin{LTR}
    %@formatter:off
        \begin{lstlisting}[style=C++Style]
              if ((i1 != 0) && (i2 != 0))
        \end{lstlisting}
    %@formatter:on
\end{LTR}

مدت کوتاهی بعد از این که من این باگ را پیدا کردم، به یکی از همکارانم گفتم. چیزی را که اتفاق افتاده بود توضیح دادم و گفتم: «من حالا تفاوت بین «و» و «وو» را می دانم».