\section{}
\paragraph{}\label{answer:60}
 برنامه‌نویس عادت بدی دارد که فایل‌ها را بعد از این که باز کرد، نمی‌بندد. خیلی زود، تعداد فایل‌های باز به ماکزیمم رسیده و سیستم به او اجازهٔ بازکردن فایل جدیدی را نمی‌دهد. در هر جا که لازم باشد باید فایل‌ها را بست:
 \begin{LTR}
    %@formatter:off
    \begin{lstlisting}[style=C++Style]
        int fd = open(cur_ent->d_name, O_RDONLY);
        if (fd < 0)
            continue; // Can't get the file so try again

        int magic; // The file's magic number
        int read_size = read(fd, &magic, sizeof(magic));

        if (read_size != sizeof(magic)) {
            close(fd); // <---- added
            continue;
        }

        if (magic == MAGIC) {
            close(fd); // <---- added
            return (cur_ent->d_name);
        }

        close(fd); // <---- added
    \end{lstlisting}
    %@formatter:on
\end{LTR}

برنامه‌نویس هم‌چنین از \lr{\texttt{opendir}} برای بازکردن یک دایرکتوری استفاده می‌کند. او هیچ وقت آن را نمی‌بندد. بنابراین یک \lr{\texttt{closedir}} لازم است.
\begin{LTR}
    %@formatter:off
    \begin{lstlisting}[style=C++Style]
        void scan_dir(
            const char dir_name[] // Directory name to use
        )
        {
            DIR *dir_info = opendir(dir_name);
            if (dir_info == NULL)
                return;
            chdir(dir_name);
            while (1) {
                char *name = next_file(dir_info);
                if (name == NULL)
                    break;
                std::cout << "Found: " << name << '\n';
            }
            closedir(dir_info); // <---- added
        }
    \end{lstlisting}
    %@formatter:on
\end{LTR}