\section{}
\paragraph{}\label{answer:92}
مشکل این است که یک سوئیچ ریسمان می‌تواند در هر زمانی رخ دهد. وقتی که \lr{\texttt{count > 0}} است، \lr{\texttt{Writer}} یک کاراکتر از بافر بیرون می‌آورد. \lr{\texttt{reader}} دو کار انجام می‌دهد:
\begin{LTR}
    %@formatter:off
    \begin{lstlisting}[style=C++Style]
        ++count; // We've got a new character
        *in_ptr = ch;// Store the character
    \end{lstlisting}
    %@formatter:on
\end{LTR}

ولی یک سوئیچ ریسمان ممکن است بین این دو مرحله رخ دهد. بنابراین، این سناریو ممکن است رخ دهد:
\begin{itemize}
    \item \lr{\texttt{reader: ++count;// We've got a new character}}
    \item سوئیچ ریسمان به \lr{\texttt{writer}}.
    \item \lr{\texttt{writer}}: بررسی این که \lr{\texttt{count > 0}} برقرار است.
    \item \lr{\texttt{writer}}: کاراکتر را بگیرد.
    \item سوئیچ ریسمان به \lr{\texttt{reader}}.
    \item \lr{\texttt{reader}}: بعد از این که \lr{\texttt{writer}}، کاراکتر را خواند، آنرا در بافر قرار دهد.
\end{itemize}

یک راه‌حل برای این کار این است که ترتیب مراحل:
\begin{LTR}
    %@formatter:off
    \begin{lstlisting}[style=C++Style]
        ++count; // We've got a new character
        *in_ptr = ch;// Store the character
    \end{lstlisting}
    %@formatter:on
\end{LTR}
را تغییر دهیم به:
\begin{LTR}
    %@formatter:off
    \begin{lstlisting}[style=C++Style]
        *in_ptr = ch;// Store the character
        ++count; // We've got a new character
    \end{lstlisting}
    %@formatter:on
\end{LTR}

اتکا به توالی دستورالعمل‌ها برای محافظت از داده‌های مشترک، مشکل و گول‌زننده است. بسیار بهتر و ساده‌تر است که به مدیر پُشته بگویید که چه زمانی دارید عبارات وقفه‌ناپذیر را انجام می‌دهید. در \lr{\texttt{pthreads}} این کار به وسیله یک قفل \lr{\texttt{mutex}} انجام‌پذیر است:
\begin{LTR}
    %@formatter:off
    \begin{lstlisting}[style=C++Style]
        pthread_mutex_lock(&buffer_mutex);

        ++count;
        *in_ptr = ch;
        ++in_ptr;

        pthread_mutex_unlock(&buffer_mutex);
    \end{lstlisting}
    %@formatter:on
\end{LTR}
