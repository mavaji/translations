\section{}
\paragraph{}\label{answer:7}
مشکل این است که \lr{\texttt{sub.cpp}}، \lr{\texttt{str}} را به صورت یک آرایه کاراکتری تعریف می‌کند (\lr{\texttt{char[]}}). عبارت \lr{\texttt{extern}} در \lr{\texttt{main.cpp}}، \lr{\texttt{str}} را بصورت اشاره‌گر کاراکتری تعریف می‌کند (\lr{\texttt{char *}}). تقریباً همیشه در \lr{\texttt{C++}}، آرایه‌های کاراکتری و اشاره‌گرهای کاراکتری قابل‌تعویض می‌باشند. این یکی از معدود حالاتی است که این قضیه صدق نمی‌کند. در این حالت، برنامه \lr{\texttt{main}} فکر می‌کند که \lr{\texttt{str}} یک اشاره‌گر کاراکتری است، لذا به آن مکان رفته و اولین چهار بایت را به عنوان آدرس می‌خواند. اولین چهار بایت برابر \lr{\texttt{"Hell"}} است که یک آدرس نمی‌باشد و بنابراین برنامه از کار می‌افتد.

همیشه \lr{\texttt{extern}}ها را در یک فایل سرآمد تعریف کنید. این سرآمد باید همیشه در ماژولی که این قلم تعریف می‌شود و در تمام ماژول‌هایی که از آن استفاده می‌کنند، \lr{\texttt{include}} شود.