\section{}
\paragraph{}\label{answer:21}
استاندارد \lr{\texttt{C++}} بیان می‌دارد که تمام اشاره‌گرها باید به یک آرایه یا آدرسی بالاتر از آن اشاره کنند. شما نمی‌توانید به آدرسی پایین‌تر از آرایه اشاره کنید. در این مثال، ما آرایه‌ای روی یک ماشین اینتل داریم. آدرس آرایه، در اشاره‌گرِ عجیب و غریب اینتل برابر \lr{\texttt{5880:0000}} است. متغیر \lr{\texttt{data\_ptr}} از \lr{\texttt{5880:001E}} شروع می‌شود. سپس تا زمانی که از \lr{\texttt{data}} بزرگ‌تر باشد مقدار آن کاهش می‌یابد. در طی کاهش یافتنِ مقدارِ آن، \lr{\texttt{data\_ptr}} به \lr{\texttt{5880:0000}} می‌رسد. این برابر آدرس آرایهٔ داده است، لذا دوباره شروع به کاهش می‌کند. (به یاد داشته باشید که در این مدلِ حافظه، فقط قسمت آدرس تغییر می‌کند). نتیجه برابر \lr{\texttt{5880:FFFE}} است.

حالا \lr{\texttt{data\_prt >= data}} ارزیابی می‌شود. ولی \lr{\texttt{data\_ptr}} اکنون خیلی بزرگ‌تر از \lr{\texttt{data}} است، لذا برنامه به کار خود ادامه می‌دهد. نتیجه این است که برنامه روی داده‌های تصادفی می‌نویسد که می‌تواند باعث از کار افتادنِ سیستم شود. ولی اگر این اتفاق نیفتد، \lr{\texttt{data\_ptr}} به \lr{\texttt{5880:0000}} تنزل می‌یابد و این فرایند دوباره شروع می‌شود.
