\section[دوی ساده]{دوی ساده \protect\LTRfootnote{\lr{Two Simple}} (راهنمایی \ref{hint:164}، جواب \ref{answer:85})}
\paragraph{}\label{prog:70}
چرا 5986 = 2 + 2؟

\begin{LTR}
    %@formatter:off
    \begin{lstlisting}[style=C++Style]
         /************************************************
         * two_plus_two -- So what is 2+2 anyway? *
         ************************************************/
         #include <stdio.h>

         int main()
         {
         	/* Result of the addition */
         	int answer = 2 + 2;

         	printf("The answer is %d\n");
         	return (0);
         }
    \end{lstlisting}
    %@formatter:on
\end{LTR}

\begin{tcolorbox}
    در پایین چک‌های بانکیِ شما، یک سری عدد وجود دارد که نشانگرِ شمارهٔ بانک و شماره حساب شما می‌باشد. کلاهبرداری، با پنج دلار، یک حساب در نیویورک باز کرد. سپس چک‌های خودش را درست کرد. آنها مثل چک‌های واقعی بودند به جز این که شمارهٔ بانک تغییر کرده بود به طوری که نشانگرِ بانکی در لوس‌آنجلس بود.

    او سپس یک حساب دیگر در نیویورک باز کرد و یک چک 10000 دلاری به عنوان موجودی اولیه در آن ریخت. چک درون دستگاهِ مرتب‌سازی اتوماتیک رفت و کامپیوتر با مشاهده شماره بانک لوس‌آنجلس، چک را به آنجا فرستاد. بانک لوس‌آنجلس متوجه شد که این چک متعلق به آنها نیست لذا آن را برای دفتر تسویه‌حساب به نیویورک پس فرستاد. چک دوباره وارد دستگاه مرتب‌سازی اتوماتیک شد، کامپیوتر شماره بانک لوس‌آنجلس را دید و آن را به آنجا فرستاد.

    چک اکنون در یک چرخه بی‌پایان بین لوس‌آنجلس و نیویورک در گردش بود. در این اثنا، کلاهبردار به بانک رفت و همه پولش را خواست. کارمند به آخرین واریز نگاه کرد و دید که مربوط به دو هفته پیش است و فکر کرد که چک تسویه شده است. چون کلاً دو روز طول می‌کشد تا یک چک نیویورک به بانک مقصد برسد. بنابراین کارمند به کلاهبردار، 10000 دلار داد و او ناپدید شد.

    چندین هفته بعد، چک آنقدر داغان شده بود که دیگر نمی‌توانست داخل دستگاه مرتب‌سازی اتوماتیک قرار بگیرد. بنابراین به طور دستی مرتب شد و به بانک مربوطه فرستاده شد.
\end{tcolorbox}