\section[بیشترین شگفتی]{بیشترین شگفتی \protect\LTRfootnote{\lr{Maximum Surprise}} (راهنمایی \ref{hint:194}، جواب \ref{answer:112})}
\paragraph{}\label{prog:25}
حلقهٔ برنامهٔ زیر برای چاپ یک پیغام خوش‌آمدگویی به اندازهٔ 10 بار طراحی شده است. ولی برنامه کار دیگری انجام می‌دهد. قضیه چیست؟
توجه: این برنامه روی کامپایلرهای \lr{GNU} و دیگر سیستم‌هایی که رهنمون‌های پیش‌پردازنده را پیاده‌سازی نمی‌کنند، کامپایل نمی‌شود.

\begin{LTR}
    %@formatter:off
        \begin{lstlisting}[style=C++Style]
             /************************************************
             * Print a bunch of greetings. *
             ************************************************/
             #include <iostream>

             #define MAX =10

             int main()
             {
             	int counter; // Current greeting

             	for (counter =MAX; counter > 0; --counter)
             		std::cout <<"Hi there\n";

             	return (0);
             }
        \end{lstlisting}
    %@formatter:on
\end{LTR}

\begin{tcolorbox}
    مرکز کامپیوتر یک دانشگاه بزرگ در یک ساختمان قدیمی قرار داشت. آنها تقریباً یک مشکلِ آزاردهنده داشتند. شب‌هنگام وقتی که اپراتور، اتاق را ترک می‌کرد، کامپیوتر ریبوت می‌شد.

    یک تکنیسین کامپیوتر فراخوانده شد و به سرعت دریافت که سیستم فقط وقتی ریبوت می‌شود که اپراتور به دستشویی می‌رود. وقتی می‌رفت آب بخورد هیچ اتفاقی نمی‌افتاد.

    یک سری از تکنیسین‌ها فراخوانده شدند تا مسأله را بررسی کنند. تجهیزات تشخیصیِ بسیاری روی کامپیوتر قرار گرفتند.

    نهایتاً ریشهٔ مشکل را پیدا کردند. زمین آن ساختمان به لوله‌های آب وصل بود. وزن اپراتور حدود 300 پوند بود و وقتی روی دستشویی می‌نشست آن را قدری خم می‌کرد و لوله‌ها جدا می‌شدند. این امر اتصال با زمین را قطع می‌کرد و باعث یک نوسان کوچک می‌شد که کامپیوتر را ریبوت می‌کرد.
\end{tcolorbox}