\section[محدودهٔ خطا]{محدودهٔ خطا \protect\LTRfootnote{\lr{Margin of Error}} (راهنمایی \ref{hint:45}، جواب \ref{answer:82})}
\paragraph{}\label{prog:78}
اگر یک کاغذ با پهنای \lr{8.5} اینچ داشته باشیم و از \lr{1} اینچ برای محدوده‌ها (\lr{0.5} اینچ برای هر طرف) استفاده کنیم، چقدر فضای قابل استفاده باقی گذاشته‌ایم؟ همه می‌دانند که جواب \lr{7.5} اینچ است. ولی این برنامه از زاویهٔ دیگری به قضایا می‌نگرد. چه شده است؟

\begin{LTR}
    %@formatter:off
    \begin{lstlisting}[style=C++Style]
         /************************************************
         * paper_size -- Find the usable width on *
         * a page. *
         ************************************************/
         #define PAPER_WIDTH 8.5; // Width of the page
         #define MARGIN 1.0; // Total margins
         // Usable space on the page
         #define USABLE PAPER_WIDTH -MARGIN;

         #include <iostream>

         int main()
         {
         	// The usable width
         	double text_width = USABLE;

         	std::cout << "Text width is " <<
         		text_width << '\n';
         	return (0);
         }
    \end{lstlisting}
    %@formatter:on
\end{LTR}

\begin{tcolorbox}
    در اوقات فراغتم، من بازی \lr{Adventure} را روی کامپیوتر شرکت نصب کرده بودم و ساعاتی را به بازی کردن می‌گذراندم. روزی مدیرم مرا به دفتر خود فراخواند.

    او پرسید: «آیا شما روی سیستم، \lr{Adventure} نصب کرده‌اید؟»

    جواب دادم: «من آن را در اوقات فراغتم بازی می‌کردم.»

    او مرا خاطرجمع کرد که: «آه، من از شما انتقاد نمی‌کنم. حقیقتِ امر این است که می‌خواهم از شما قدردانی کنم. از موقعی که این پروژه شروع شده است، بیل (مسئول بازاریابی) هر روز این جا می‌آید. هر روز او وارد می‌شود، با نرم‌افزار کار می‌کند و اصرار به برخی تغییرات در برنامه دارد. ولی در هفتهٔ گذشته، تمامِ وقتِ خود را به بازی با \lr{Adventure} گذراند و وقتی برایش نماند که تقاضای تغییرات بکند. من می‌خواستم از شما به خاطر اینکه شرِ او را از سر من کم کرده‌اید، تشکر کنم.
\end{tcolorbox}