\section[بیهوده]{بیهوده \protect\LTRfootnote{\lr{Pointless}} (راهنمایی \ref{hint:298}، جواب \ref{answer:78})}
\paragraph{}\label{prog:73}
انواع داده \lr{\texttt{sam}} و \lr{\texttt{joe}} چیستند؟

\begin{LTR}
    %@formatter:off
    \begin{lstlisting}[style=C++Style]
         /************************************************
         * Toy program that declares two variables *
         ************************************************
         #define CHAR_PTR char *

         int main()
         {
         	CHAR_PTR sam, joe;

         	return (0);
         }
    \end{lstlisting}
    %@formatter:on
\end{LTR}

\begin{tcolorbox}
    من روی اولین چاقوی فشار آب کار می‌کردم. آن دستگاهی بود که کفِ کفش‌های تنیس را با فشار بالایی از آب می‌بُرید. از آنجا که آن، اولین نمونه از نوع خود بود، زمان بسیاری را برای میزان‌سازیِ آن صرف کردیم، حدود یک سال. ما با شرکت سازندهٔ کفشِ تنیس که قرار بود آن را بخرد، قراردادی داشتیم. آنها به ما موادِ خامِ رایگان برای آزمایش می‌دادند اگر که تکه‌های بریده‌شده را به آنها برمی‌گرداندیم.

    ما حدود یک سال آزمایش کردیم. چون می‌خواستیم همیشه نتایجِ همسان داشته باشیم، تقریباً همیشه از یک آزمون استفاده می‌کردیم: نُهِ راست. ما با وظیفه‌شناسی، قطعات بریده‌شده را بسته‌بندی می‌کردیم و به سازندهٔ کفش‌های تنیس باز می‌گرداندیم که آنها بتوانند از آن تکه‌ها کفش بسازند یا حداقل ما این طور فکر می‌کردیم.

    حدود یک هفته قبل از این که دستگاه را تحویل دهیم، کسی از کارخانهٔ کفش تنیس با ما تماس گرفت.

    کارخانهٔ کفش تنیس: «آیا شما همان افرادی هستید که مرتب به ما نُهِ راست را می‌فرستید؟»\\
    ما: «بله»\\
    - بالاخره شما را پیدا کردم. برای یک سال دنبالتان بودم. هیچ اطلاعاتی از خرید قطعات بریده‌شده وجود نداشت و یافتن شما    خیلی سخت بود.\\
    - مشکلی پیش آمده؟\\
    - بله. آیا متوجه نشده‌اید که شما 10000 راستِ نُه به ما تحویل داده‌اید و هیچ لنگهٔ چپی نفرستاده‌اید؟
\end{tcolorbox}