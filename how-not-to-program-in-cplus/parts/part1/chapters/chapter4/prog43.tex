\section[بی‌اساس]{بی‌اساس \protect\LTRfootnote{\lr{Baseless}} (راهنمایی \ref{hint:330}، جواب \ref{answer:58})}
\paragraph{}\label{prog:43}
می‌دانیم که \lr{\texttt{2}} یک \lr{\texttt{int}} است. پس چرا \lr{\texttt{C++}} فکر می‌کند که آن، یک عدد اعشاری است و تابعِ غلط را فراخوانی می‌کند؟

\begin{LTR}
    %@formatter:off
    \begin{lstlisting}[style=C++Style]
         /************************************************
         * demonstrate the use of derived classes. *
         ************************************************/
         #include <iostream>

         /************************************************
         * base -- A sample base class. *
         * Prints various values. *
         ************************************************/
         class base
         {
         	// Constructor defaults
         	// Destructor defaults
         	// Copy constructor defaults
         	// Assignment operator defaults
         	public:
         		// Print a floating point number
         		void print_it(
         			float value // The value to print
         		)
         		{
         			std::cout <<
         				"Base (float=" << value << ")\n";
         		}
         		// Print an integer value
         		void print_it(
         			int value // The value to print
         		)
         		{
         			std::cout <<
         				"Base (int=" << value << ")\n";
         		}
         };

         class der
         {
         	// Constructor defaults
         	// Destructor defaults
         	// Copy constructor defaults
         	// Assignment operator defaults
         	public:
         		// Print a floating point number
         		void print_it(
         			float value // The value to print
         		)
         		{
         			std::cout <<
         				"Der (float=" << value << ")\n";
         		}
         };

         int main()
         {
         	der a_var; // A class to play with

         	// Print a value using der::print_it(float)
         	a_var.print_it(1.0);

         	// Print a value using base::print_it(int)
         	a_var.print_it(2);
         	return (0);
         }
    \end{lstlisting}
    %@formatter:on
\end{LTR}

\begin{tcolorbox}
    نسخهٔ اصلی دستور \lr{\texttt{mt}} یونیکس یک پیغام خطای غیرعادی داشت که وقتی که نمی‌توانست دستوری را بفهمد، ظاهر می‌شد:
    \LTR\noindent
    \lr{\texttt{mt –f /dev/rst8 funny}}\\
    \lr{\texttt{mt: Can’t grok funny}}
    \RTL
    برای کسانی که با «غریبه در غربت» نوشته روبرت هاینلین آشنا نیستند، باید بگویم که grok معادل مریخی فهمیدن است.

    این اصطلاح به کشورهای دیگر راه نیافت. یک برنامه‌نویسِ آلمانی با سادگی تمام رفت که معنی grok را در لغتنامهٔ انگلیسی-آلمانی بیابد.
\end{tcolorbox}