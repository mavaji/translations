\section[موردِ آرایهٔ ناپدید شونده]{موردِ آرایهٔ ناپدید شونده \protect\LTRfootnote{\lr{The Case of the Disappearing Array}} (راهنمایی \ref{hint:189}، جواب \ref{answer:59})}
\paragraph{}\label{prog:82}
ما یک کلاس ساده آرایه و یک روتین ساده‌تر برای آزمون داریم. ولی به طریقی حافظه خراب می‌شود.

\begin{LTR}
    %@formatter:off
        \begin{lstlisting}[style=C++Style]
             /*************************************************
             * var_array -- Test variable length array *
             * class. *
             *************************************************/
             #include <memory.h>

             /*************************************************
             * var_array -- Variable length array *
             * *
             * Member functions: *
             * operator [] -- Return a reference to *
             * the item in the array. *
             *************************************************/

             class var_array
             {
             	private:
             		int *data; // The data
             		const int size; // The size of the data
             	public:
             		// Create the var_array
             		var_array(const int _size):
             		size(_size)
             		{
             			data = new int[size];
             			memset(data, '\0',
             			size * sizeof(int));
             		}
             		// Destroy the var_array
             		~var_array(void) {
             			delete []data;
             		}
             	public:
             		// Get an item in the array
             		int &operator [] (
             			// Index into the array
             			const unsigned index
             		)
             		{
             			return (data[index]);
             		}
             };

             /************************************************
             * store_it -- Store data in the var_array *
             ************************************************/
             static void store_it(
             	// Array to use for storage
             	var_array test_array
             )
             {
             	test_array[1] = 1;
             	test_array[3] = 3;
             	test_array[5] = 5;
             	test_array[7] = 7;
             }
             int main()
             {
             	var_array test_array(30);

             	store_it(test_array);
             	return (0);
             }
        \end{lstlisting}
    %@formatter:on
\end{LTR}

\begin{tcolorbox}
    \centering
    قانون مستندسازیِ اُلین

    \raggedright
    90 درصدِ اوقات، مستندسازی وجود ندارد. در 10 درصدِ باقیمانده، 9 درصدِ اوقات، مستندسازی برای یک نسخهٔ اولیهٔ نرم‌افزار است و لذا بدون استفاده می‌باشد. در 1 درصدِ مواقع، شما مستندسازی و نسخهٔ درستی از آن را دارید ولی به ژاپنی نوشته شده است.

    من این طنز را به رفیقی در موتورولا گفتم و او برای چند دقیقه خندید. سپس یک راهنمای فرترنِ هیتاچی را بیرون آورد که به ژاپنی نوشته شده بود.
\end{tcolorbox}