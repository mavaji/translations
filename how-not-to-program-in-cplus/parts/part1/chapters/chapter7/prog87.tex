\section[استثنای استثنایی]{استثنای استثنایی \protect\LTRfootnote{\lr{Exceptional Exception}} (راهنمایی \ref{hint:110}، جواب \ref{answer:55})}
\paragraph{}\label{prog:87}
این کلاسِ پشته طراحی شده است تا مستحکم‌تر\LTRfootnote{\lr{Robust}} باشد و اگر چیز اشتباهی در پشته پیش آمد، استثنایی را ایجاد کند\LTRfootnote{\lr{Throw an Exception}}. ولی باز هم این برنامه قطع می‌شود و درست کار نمی‌کند. چرا؟

\begin{LTR}
    %@formatter:off
        \begin{lstlisting}[style=C++Style]
             /************************************************
             * stack_test -- Yet another testing of a *
             * stack class. *
             ************************************************/
             #include <iostream>

             /************************************************
             * problem -- Class to hold a "problem". Used *
             * for exception throwing and catching. *
             * *
             * Holds a single string which describes the *
             * error. *
             ************************************************/
             class problem
             {
             	public:
             		// The reason for the exception
             		char *what;

             		// Constructor.
             		// Create stack with messages.
             		problem(char *_what):what(_what){}
             };

             // Max data we put in a stack
             // (private to the stack class)
             const int MAX_DATA = 100;
             /************************************************
             * stack -- Classic stack. *
             * *
             * Member functions: *
             * push -- Push an item on the stack. *
             * pop -- Remove an item from the stack. *
             * *
             * Exceptions: *
             * Pushing too much data on a stack or *
             * removing data from an empty stack *
             * causes an exception of the "problem" *
             * class to be thrown. *
             * *
             * Also if you don't empty a stack *
             * before you're finished, an exception *
             * is thrown. *
             *************************************************/
             class stack {
             	private:
             		// The stack's data
             		int data[MAX_DATA];

             		// Number of elements
             		// currently in the stack
             		int count;

             	public:
             		// Constructor
             		stack(void) : count(0) {};

             		// Destructor -- Check for non
             		~stack(void)
             		{
             			if (count != 0)
             			{
             				throw(
             					problem("Stack not empty"));
             			}
             		}

             		// Push an item on the stack
             		void push(
             			const int what // Item to store
             		)
             		{
             			data[count] = what;
             			++count;
             		}
             		// Remove an item from the stack
             		int pop(void)
             		{
             			if (count == 0)
             				throw(
             					problem("Stack underflow"));
             			--count;
             			return (data[count]);
             		}
             };

             /************************************************
             * push_three -- Push three items onto a stack *
             * *
             * Exceptions: *
             * If i3 is less than zero, a "problem" *
             * class exception is thrown. *
             ************************************************/
             static void push_three(
             	const int i1, // First value to push
             	const int i2, // Second value to push
             	const int i3 // Third value to push
             )
             {
             	// Stack on which to push things
             	stack a_stack;

             	a_stack.push(i1);
             	a_stack.push(i2);
             	a_stack.push(i3);
             	If (i3 < 0)
             	throw (problem("Bad data"));
             }

             int main(void)
             {
             	try {
             		push_three(1, 3, -5);
             	}
             	catch (problem &info) {

             		std::cout << "Exception caught: " <<
             			info.what << std::endl;

             		exit (8);
             	}
             	catch (...) {
             		std::cout <<
             			"Caught strange exception " <<
             			std::endl;

             		exit (9);
             	}
             	std::cout << "Normal exit" << std::endl;
             	return (0);
             }
        \end{lstlisting}
    %@formatter:on
\end{LTR}