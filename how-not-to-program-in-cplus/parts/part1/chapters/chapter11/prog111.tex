\section[دست‌اندازی در مسیر مسابقه]{دست‌اندازی در مسیر مسابقه \protect\LTRfootnote{\lr{A bumb on the Race Track}} (راهنمایی \ref{hint:222}، جواب \ref{answer:92})}
\paragraph{}\label{prog:111}
این برنامه، دو ریسمان\LTRfootnote{\lr{Thread}} را آغاز می‌کند. یکی داده را درون یک بافِر می‌ریزد و دیگری داده را درون یک فایل می‌ریزد. ولی برخی اوقات، داده خراب می‌شود. چرا؟

\begin{LTR}
    %@formatter:off
        \begin{lstlisting}[style=C++Style]
             /***********************************************
             * Starts two threads *
             * *
             * 1) Reads data from /dev/input and puts *
             * it into a buffer. *
             * *
             * 2) Takes data from the buffer and *
             * writes the data to /dev/output.*
             ***********************************************/
             #include <cstdio>
             #include <stdlib.h>
             #include <pthread.h>
             #include <unistd.h>
             #include <sys/fcntl.h>

             static const int BUF_SIZE = 1024; // Buffer size
             static char buffer[BUF_SIZE]; // The data buffer

             // Pointer to end of buffer
             static char *end_ptr = buffer + BUF_SIZE;

             // Next character read goes here
             static char *in_ptr = buffer;

             // Next character written comes from here
             static char *out_ptr = buffer;

             static int count = 0; // Number of characters in the buffer

             /***********************************************
             * reader -- Read data and put it in the global *
             * variable buffer. When data is *
             * installed the variable count is *
             * increment and the buffer pointer *
             * advanced. *
             ************************************************/
             static void *reader(void *) {
             	// File we are reading
             	int in_fd = open("/dev/input", 0_RDONLY);

             	while (1) {
             		char ch; // Character we just got

             		while (count >= BUF_SIZE)
             			sleep(1);

             		read(in_fd, &ch, 1);

             		++count;
             		*in_ptr = ch;
             		++in_ptr;

             		if (in_ptr == end_ptr)
             		in_ptr = buffer;
             	}
             }

             /***********************************************
             * writer -- Write data from the buffer to *
             * the output device. Gets the data *
             * from the global buffer. Global variable*
             * count is decrement for each character *
             * taken from the buffer and the buffer *
             * pointer advanced. *
             ***********************************************/
             static void writer(void)
             {
             	// Device to write to
             	int out_fd = open("/dev/output", 0_RDONLY);

             	while (1) {
             		char ch; // Character to transfer

             		while (count <= 0)
             			sleep(1);

             		ch = *out_ptr;

             		--count;
             		++out_ptr;

             		if (out_ptr == end_ptr)
             			out_ptr = buffer;

             		write(out_fd, &ch, 1);
             	}
             }

             int main() {
             	int status; /* Status of last system call */

             	/* Information on the status thread */
             	pthread_t reader_thread;

             	status = pthread_create(&reader_hread, NULL, reader, NULL);

             	if (status != 0) {
             		perror("ERROR: Thread create failed:\n ");
             		exit (8);
             	}

             	writer();
             	return (0);
             }
        \end{lstlisting}
    %@formatter:on
\end{LTR}

\begin{tcolorbox}
    در طیِ سالیان، نصب‌کننده‌های سیستم راه‌های گوناگونی را برای کابل‌کِشی سقف‌های کاذب ایجاد کرده‌اند. یکی از مبتکرانه‌ترینِ آنها، روش «سگ کوچک» است. یک نفر سگ کوچکی را می‌گیرد، رشته‌ای به گردنش می‌بندد و سگ را درون سقف قرار می‌دهد. سپس صاحب به سمت سوراخی می‌رود که می‌خواهند کابل از آنجا بیرون بیاید و سگ را صدا می‌زند. سگ به طرف صاحبش می‌دود. آنها یک کابل به رشته می‌بندند و از این طریق، کابل‌کِشی را انجام می‌دهند.
\end{tcolorbox}