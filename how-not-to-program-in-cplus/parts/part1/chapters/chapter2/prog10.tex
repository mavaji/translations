\section[تقسیمِ نه چندان بزرگ]{تقسیمِ نه چندان بزرگ \protect\LTRfootnote{\lr{The Not-So-Great-Divide}} (راهنمایی \ref{hint:292}، جواب \ref{answer:27})}
این برنامهٔ ساده‌ای است که نشان می‌دهد چند رقم بامعنی برای اعداد اعشاری استفاده می‌شود. ایدهٔ آن ساده است: یک کسر با دور گردش مثل \lr{1/3 = 0.3333333} را در نظر بگیر، آن را نمایش بده و ببین چند رقم آن نمایش داده می‌شود.

با این حال، نتایجِ برنامه، برنامه‌نویس را گیج کرد. او می‌دانست که کامپیوتر نمی‌تواند این قدر احمق باشد، پس چه اتفاقی افتاده؟

\begin{LTR}
    %@formatter:off
        \begin{lstlisting}[style=C++Style]
             /************************************************
             * divide -- Program to figure out how many *
             * digits are printed in floating point *
             * by print 1/3 or 0.333333. *
             ************************************************/
             #include <iostream>

             int main()
             {
             float result; // Result of the divide

             result = 1/3; // Assign result something

             std::cout << "Result is " << result << '\n';
             return (0);
             }
        \end{lstlisting}
    %@formatter:on
\end{LTR}

\begin{tcolorbox}
    یک هواشناس باید در کامپیوترِ ادارهٔ هواشناسی، مقدارِ باران را بر حسب اینچ وارد می‌کرد. افراد آنجا عادت کرده بودند که با صدم اینچ کار کنند، لذا وقتی از آنها پرسیده می‌شد که امروز چقدر باران آمده، جواب \lr{50} به معنی \lr{50/100} اینچ یا نصف اینچ می‌بود.

    با این حال برای وارد کردن این در کامپیوتر باید نوشته می‌شد \lr{0.50}. یکی از افراد این مطلب را فراموش کرد و میزانِ بارشِ باران را به صورت \lr{50} وارد کرد. حالا \lr{50} اینچ، بارانِ زیادی است. میزانِ فوق‌العاده‌ای باران است. کامپیوتر خطا گرفت و پیغام مناسب را نمایش داد:

    یک کشتی بساز و از هر کدام از جانداران جفتی بردار...
\end{tcolorbox}