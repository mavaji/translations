\section[برنامه‌نویسی فریب‌آمیز]{برنامه‌نویسی فریب‌آمیز \protect\LTRfootnote{\lr{Shifty Programming}} (راهنمایی \ref{hint:266}، جواب \ref{answer:49})}
\paragraph{}\label{prog:14}
برنامه‌نویس می‌داند که شیفت دادن به چپ مانند ضرب در توانی از 2 است. به عبارت دیگر:
\LTR
\noindent\lr{\texttt{x << 1}} \rl{برابر است با}  \lr{\texttt{x * 2 (2 = 2\textsuperscript{1})}}\\
\lr{\texttt{x << 2}} \rl{برابر است با}  \lr{\texttt{x * 4 (4 = 2\textsuperscript{2})}}\\
\lr{\texttt{x << 3}} \rl{برابر است با}  \lr{\texttt{x * 8 (8 = 2\textsuperscript{3})}}
\RTL
برنامه‌نویس از این ترفند برای انجام یک محاسبهٔ سریع استفاده می‌کند ولی یک چیز اشتباه است:

\begin{LTR}
    %@formatter:off
        \begin{lstlisting}[style=C++Style]
             /************ *********** ************ **********
             * Simple syntax testing. *
             ************************************************/
             #include <iostream>

             int main(void)
             {
             	int x,y; // Two numbers

             	x = 1;

             	y = x<<2 + 1; // x<<2 = 4 so y = 4+1 = 5
             	std::cout << "Y=" << y << std::endl;
             	return (0);
             }
        \end{lstlisting}
    %@formatter:on
\end{LTR}

\begin{tcolorbox}
    یک هکر مأمور شد تا برنامه‌ای بنویسد که یک ماشین‌حسابِ چهارکاره را شبیه‌سازی کند. این برنامه باید عملِ جمع، تفریق، ضرب و تقسیم را انجام می‌داد. با این حال مشخص نشده بود که چه نوع عددی باید استفاده شود. بنابراین برنامهٔ هکر با اعداد رومی کار می‌کرد \lr{(IV + III = VII)}. به یک دفترچه راهنمای کاربر نیز نیاز بود ولی زبان آن هم مشخص نشده بود. بنابراین برنامه‌نویس یک راهنمای مفصل به زبان لاتین تهیه کرد.
\end{tcolorbox}