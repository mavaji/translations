\section[نه خوب نه بد]{نه خوب نه بد \protect\LTRfootnote{\lr{Nothing Special}} (راهنمایی \ref{hint:175}، جواب \ref{answer:81})}
\paragraph{}\label{prog:104}
هدف عبارت \lr{\texttt{if}} در زیربرنامهٔ زیر چیست؟ به نظر می‌رسد که کاملاً بی‌مورد باشد.

\begin{LTR}
    %@formatter:off
        \begin{lstlisting}[style=C++Style]
             /************************************************
             * sum_file -- Sum the first 1000 integers in *
             * a file. *
             ************************************************/
             #include <iostream>
             #include <fstream>
             /************************************************
             * get_data -- Get an integer from a file. *
             * *
             * Returns: The integer gotten from the file *
             ************************************************/
             int get_data(
             	// The file containing the input
             	std::istream &in_file
             ) {
             	int data; // The data we just read
             	static volatile int seq = 0; // Data sequence number

             	++seq;
             	if (seq == 500)
             		seq = seq; // What's this for?

             	in_file.read(&data, sizeof(data));
             	return (data);
             }

             int main() {
             	int i; // Data index
             	int sum = 0; // Sum of the data so far

             	// The input file
             	std::ifstream in_file("file.in");

             	for (i = 0; i < 1000; ++i) {
             		sum = sum + get_data(in_file);
             	}
             	std::cout << "Sum is " << sum << '\n';
             	return (0);
             }
        \end{lstlisting}
    %@formatter:on
\end{LTR}
