\section{}
\paragraph{}\label{answer:85}
عبارت
\begin{LTR}
    %@formatter:off
        \begin{lstlisting}[style=C++Style]
            11 printf("The answer is %d\n");
        \end{lstlisting}
    %@formatter:on
\end{LTR}
به \lr{\texttt{C}} می‌گوید که یک عدد صحیح را چاپ کند، ولی آن عدد صحیح را فراهم نمی‌آورد. تابع \lr{\texttt{printf}} این را نمی‌داند، لذا اولین عدد را از پُشته بیرون می‌آورد (یک عدد تصادفی) و آن را چاپ می‌کند. چیزی که برنامه‌نویس می‌بایستی بنویسد این است:
\begin{LTR}
    %@formatter:off
        \begin{lstlisting}[style=C++Style]
            printf("The answer is %d\n", answer);
        \end{lstlisting}
    %@formatter:on
\end{LTR}
