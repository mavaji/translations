\section{}
\paragraph{}\label{answer:113}
فاصله‌ای که بعد از اسم \lr{\texttt{DOUBLE}} وجود دارد، این ماکرو را تبدیل به یک ماکروی سادهٔ جابجاییِ متن می‌سازد. بنابراین،
\begin{LTR}
    %@formatter:off
    \begin{lstlisting}[style=C++Style]
        #define DOUBLE (value) ((value) + (value))
    \end{lstlisting}
    %@formatter:on
\end{LTR}

باعث می‌شود که \lr{\texttt{DOUBLE}} تعویض شود با:
\begin{LTR}
    %@formatter:off
    \begin{lstlisting}[style=C++Style]
        (value) ((value) + (value))
    \end{lstlisting}
    %@formatter:on
\end{LTR}

و این بدان معناست که خط:
\begin{LTR}
    %@formatter:off
    \begin{lstlisting}[style=C++Style]
        std::cout << "Twice " << counter << " is " <<
            DOUBLE(counter) << '\n';
    \end{lstlisting}
    %@formatter:on
\end{LTR}

بدین صورت در می‌آید:
\begin{LTR}
    %@formatter:off
    \begin{lstlisting}[style=C++Style]
        std::cout << "Twice " << counter << " is " <<
            (value) ((value) + (value)) (counter) << '\n';
    \end{lstlisting}
    %@formatter:on
\end{LTR}

(با اضافه کردن توگذاری)

راه‌حل: \lr{\texttt{DOUBLE}} را بدین صورت تعریف کنید:
\begin{LTR}
    %@formatter:off
    \begin{lstlisting}[style=C++Style]
        #define DOUBLE(value) ((value) + (value))
    \end{lstlisting}
    %@formatter:on
\end{LTR}

تا آنجا که ممکن است، به جای ماکروهای پارامتردار از توابع \lr{\texttt{inline}} استفاده کنید. مثال:
\begin{LTR}
    %@formatter:off
    \begin{lstlisting}[style=C++Style]
        inline DOUBLE(const int value) {
            return (value + value);
        }
    \end{lstlisting}
    %@formatter:on
\end{LTR}

