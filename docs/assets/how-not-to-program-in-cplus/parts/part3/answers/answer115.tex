\section{}
\paragraph{}\label{answer:115}
من همیشه سعی می‌کنم مطمئن شوم که متغیر \lr{\texttt{data}} را قبل از این که بازنویسی کنم، \lr{\texttt{delete}} کنم. بنابراین مشکل حافظه نخواهم داشت. من حتی در کد زیر نیز آن را پاک می‌کنم:
\begin{LTR}
    %@formatter:off
    \begin{lstlisting}[style=C++Style]
        34 // Copy constructor
        35 v_string(const v_string &old)
        36 {
        37 	if (data != NULL)
        38 	{
        39 		delete[] data;
        40 		data = NULL;
        41 	}
        42	 data = strdup(old.data);
        43 }
    \end{lstlisting}
    %@formatter:on
\end{LTR}

این، تابع سازنده کپی است. اولین کاری که انجام می‌دهد این است که ببیند آیا \lr{\texttt{data}} چیزی درون خود دارد یا نه، و اگر داشت، آن را \lr{\texttt{delete}} می‌کند. ولی \lr{\texttt{data}} چه چیزی می‌تواند در خود داشته باشد؟ ما فقط کلاس را ایجاد کردیم و آن را مقداردهی اولیه نکرده‌ایم. بنابراین داریم یک اشاره‌گر تصادفی را پاک می‌کنیم و در نتیجه، برنامه از کار می‌افتد. تابع سازنده‌ای که به درستی نوشته شده باشد، این است:
\begin{LTR}
    %@formatter:off
    \begin{lstlisting}[style=C++Style]
        34 // Copy constructor
        35 v_string(const v_string &old):
        36 	data(strdup(old.data))
        37 {}
    \end{lstlisting}
    %@formatter:on
\end{LTR}

