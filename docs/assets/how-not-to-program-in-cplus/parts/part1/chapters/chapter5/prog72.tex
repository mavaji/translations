\section[بینش بی‌انتها]{بینش بی‌انتها \protect\LTRfootnote{\lr{No End in Sight}} (راهنمایی \ref{hint:15}، جواب \ref{answer:63})}
\paragraph{}\label{prog:72}
این برنامهٔ ساده برای کپی کردنِ ورودیِ استاندارد به خروجیِ استاندارد است. این یکی از اولین برنامه‌های مرتبط با \lr{I/O} است که یک دانشجو می‌نویسد.

\begin{LTR}
    %@formatter:off
    \begin{lstlisting}[style=C++Style]
         /************************************************
         * copy -- Copy stdin to stdout. *
         ************************************************/
         #include <stdio.h>

         int main()
         {
         	// Character to copy
         	char ch;

         	while ((ch = getchar()) != EOF)
         	{
         		putchar(ch);
         	}
         	return (0);
         }
    \end{lstlisting}
    %@formatter:on
\end{LTR}

\begin{tcolorbox}
    دو راه برای طراحی نرم‌افزار وجود دارد. یک راه این است که آن را آن قدر ساده سازیم که آشکارا هیچ ایرادی نداشته باشد و راه دیگر این است که آن را آن قدر پیچیده سازیم که هیچ ایرادِ آشکاری نداشته باشد.

    \LTR
    \rl{س. ا. ر. هوار}
\end{tcolorbox}