\section[بر سر فاصله چه آمد؟]{بر سر فاصله چه آمد؟ \protect\LTRfootnote{\lr{Gotta Have My Space}} (راهنمایی \ref{hint:247}، جواب \ref{answer:23})}
این یک برنامهٔ کوتاهِ آزمایشی است که توسط شخصی در اولین روزهای برنامه‌نویسی‌اش نوشته شده است. این برنامه برای نمایش یک جواب ساده طراحی شده است. ولی کارها درست پیش نمی‌روند.

\begin{LTR}
    %@formatter:off
        \begin{lstlisting}[style=C++Style]
             /************************************************
             * Double a number. *
             ************************************************/
             #include <iostream>

             int main(void)
             {
             	int number; // A number to double

             	std::cout << "Enter a number:";
             	std::cin >> number;

            	 std::cout << "Twice" << number << "is" <<
             		(number * 2) << '\n';
             	return (0);
             }
        \end{lstlisting}
    %@formatter:on
\end{LTR}

\begin{tcolorbox}
    من مدتی برنامه‌نویسی تدریس می‌کردم. در آن زمان چیز زیادی دربارهٔ تدریس نمی‌دانستم و تعیینِ میزانِ تکلیف برای دانش‌آموزان برایم سخت بود. یک بار توسط پلیس فورت‌ورث متوقف شدم چون تکلیف‌هایم خیلی سخت بود. ماجرایی واقعی.

    داشتم در خیابان‌های فورت‌ورث رانندگی می‌کردم و پشت یک چراغ‌قرمز توقف کردم. یک ماشین پلیس کنار من ایستاد. من به افسر نگاه کردم. او لحظه‌ای به من نگاه کرد و اشاره کرد که شیشهٔ ماشینم را پایین بیاورم. اقرار می‌کنم که کمی نگران بودم. تازه، من داشتم با یک شورولت 58 تعمیرنشده که اگزوز آن سه بار افتاده بود حرکت می‌کردم.

    شیشه را پایین آوردم و او به من فریاد زد که «استیو، تکلیف‌های این هفته‌ات خیلی مشکل‌اند».

    آن موقعی بود که فهمیدم یکی از دانش‌آموزانم برای اداره پلیس فورت‌ورث کار می‌کرده است. نیازی به گفتن نیست که من یک هفته اضافه به دانش‌آموزان وقت دادم تا تکلیف‌های خود را تحویل دهند.
\end{tcolorbox}