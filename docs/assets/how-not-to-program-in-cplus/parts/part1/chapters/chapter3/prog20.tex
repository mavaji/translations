\section[ساده تر از حد انتظار]{ساده تر از حد انتظار \protect\LTRfootnote{\lr{Simpler Than Expected}} (راهنمایی \ref{hint:193}، جواب \ref{answer:34})}
\paragraph{}\label{prog:20}
برنامهٔ زیر قرار است یک لیست از مجذور اعداد 1 تا 10 تولید کند. یک لیست از مجذورها تولید می‌کند ولی آن چیزی نیست که برنامه‌نویس انتظار داشته است.

\begin{LTR}
    %@formatter:off
        \begin{lstlisting}[style=C++Style]
             /************************************************
             * Print out the square of the numbers *
             * from 1 to 10 *
             ************************************************/
             #include <iostream>

             int main()
             {
             	int index; /* Index into the table */

             	for (index = 1; index <= 10; ++index);
             		std::cout << index << " squared " <<
             			(index * index) << '\n';

             	return (0);
             }
        \end{lstlisting}
    %@formatter:on
\end{LTR}

\begin{tcolorbox}
    برنامه‌نویسان واقعی به \lr{PL/I} برنامه نمی‌نویسند. \lr{PL/I} برای برنامه‌نویسانی است که نمی‌دانند به کوبول برنامه بنویسند یا به فرترن.

    برنامه‌نویسان واقعی وقتی \lr{Adventure} یا \lr{Rogue} بازی می‌کنند، بهتر فکر می‌کنند.

    برنامه‌نویسان واقعی به فرترن برنامه نمی‌نویسند. فرترن برای فرکانس فشار لوله و محاسبات کریستالوگرافی است. فرترن برای مهندسان ابلهی است که جوراب‌های تمیز و سفید می‌پوشند.

    مهندسان نرم‌افزارِ واقعی، برنامه‌ها را دیباگ نمی‌کنند. آنها درستی برنامه را تشخیص می‌دهند. این فرایند مستلزم اجرای چیزی روی کامپیوتر نیست به جز احیاناً یک بستهٔ کمکی برای تشخیص درستی.

    مهندسان نرم‌افزارِ واقعی ایدهٔ یک سخت‌افزار پیچیده و ثقیل در مسافتی دور که می‌تواند هر لحظه از کار بیفتد را دوست ندارند. آنها به افراد سخت‌افزاری بسیار بی‌اعتماد هستند و امیدوارند چنین سیستم‌هایی به صورت مجازی در هر سطحی وجود داشته باشد. آنها به کامپیوترهای شخصی علاقه دارند مگر اینکه بخواهند بستهٔ نرم‌افزاری کمکی برای تشخیص درستی را اجرا کنند.
\end{tcolorbox}